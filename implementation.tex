\section{設計}
        今回測定したサンプルは3つである。作成時期順に列挙すると、基本形となるrf-SQUIDを共振器間に配置したLCC。そしてrf-SQUIDと共振器間の結合部にジョセフソン接合を導入したJLCC、最後に接合部をミアンダインダクタンスに変更したMLCCである。まずは基本型の構造は先行論文がとっている手法と全く同じである。
        \begin{figure}[H]
            \centering
            \includegraphics[width=14cm]{sample.pdf}
            \caption{サンプル図}
        \end{figure}
        上図は作成サンプルの回路図を模式的に表したものである。共振器とrf-SQUID間は線路に流れる電流により生じる磁場を介して結合している。模式図のように素子間が完全に接地した状態の結合のことをGalvanic結合と呼ぶ。この構造では、素子間の相互インダクタンスが接地部分の自己インダクタンスとほぼ等しくなるため非常に強力な相互インダクタンスを得ることができる。

        設計のポイントは①rf-SQUID-共振器間の相互インダクタンス増強と②rf-SQUIDスクリーニングパラメータの精密な製造である。この2つのパラメータは結合強度を向上する上で非常に重要な因子となる。
        この2つのパラメータが結合強度向上に大きく寄与するということを基本型であるrf-SQUIDの遷移スペクトルの計算を通して論拠する。
    \subsection{遷移スペクトル計算}
        基本型の回路のハミルトニアンは
        \begin{equation}
            \hat{\mathcal{H}}=\hbar\left(\hat{a}^{\dagger }\ \hat{b}^{\dagger }\right)\left(\begin{array}{cc}
            \tilde{\omega}_{a} & g(\Phi ) \\
            g(\Phi ) & \hat{\omega}_{b}
            \end{array}\right)\left(\begin{array}{l}
            \hat{a} \\
            \hat{b}
            \end{array}\right)
        \end{equation}
        \begin{equation}
            = \hbar \hat{\omega}_{a} \hat{a}^{\dagger} \hat{a}+\hbar \hat{\omega}_{b} \hat{b}^{\dagger} \hat{b}+\hbar g(\Phi)\left(\hat{a}^{\dagger}\hat{b}+\hat{a} \vec{b}^{+}\right)
        \end{equation}

        である。式の前項2つが左右各共振器の調和振動子ポテンシャルである。最終項が結合項であり、この結合項に依存して2つの共振器が反発することを示す。各共振器の生成消滅演算子を
        \begin{equation}
            \hat{c}_{\pm}=\frac{\hat{a} \pm \hat{b}}{\sqrt{2}} \quad \hat{c}_{+}^{\dagger}=\frac{\hat{a}^{\dagger} \pm \hat{b}^{\dagger}}{\sqrt{2}}
        \end{equation}

        と置き換える。ここで演算子$\hat{c}_{\pm}$は各共振器の状態が混合した状態である。このように変換を行うと最終的に得られる状態は
        \begin{equation}
            \hat{H}=\hbar \Omega_+\hat{c}^{\dagger}_+\hat{c}_+ + \hbar \Omega_-\hat{c}^{\dagger}_-\hat{c}_- + \hbar \Delta\left(\hat{c}^{\dagger}_+ \hat{c}_- +\hat{c}^{\dagger}_- \hat{c}_{+}\right)
        \end{equation}

        \begin{equation}
            =\hbar\left(\begin{array}{cc}
            \hat{c}^{\dagger}_{+} & \hat{c}^{\dagger}_-
            \end{array}\right)\left(\begin{array}{cc}
            \Omega_{+} & \Delta \\
            \Delta & \Omega_{-}
            \end{array}\right)\left(\begin{array}{l}
            \hat{c}_{+} \\
            \hat{c}_{-}
            \end{array}\right)
        \end{equation}

        となる。この時$\Omega_+ = (\omega_a+\omega_b)/2 + g$、$\Omega_- = (\omega_a+\omega_b)/2 - g$、$\Delta = (\omega_a-\omega_b)/2$である。これより、2つの独立な調和振動子系は結合項$g$によって新たな基準モードで書き表すことができる。この新たな基底同士は元々の共振周波数の離調$(\omega_a-\omega_b)/2$で書き表すことができる。また変換前の基底、つまり左右の調和振動系の結合項$g$は新基底において新固有値の差分に現れる。これを図示すると以下のようになる。
        \begin{figure}[H]
            \centering
            \includegraphics[width=16cm]{standard_eigen.pdf}
            \caption{遷移周波数図}
        \end{figure}
        つまり、測定において得られた2つの基準モードの差分を計算することにより元のハミルトニアンの結合強度gを見積もることができる。以下に上図に対応する外部磁束に対応する結合強度の図をプロットする。
        \begin{figure}[H]
            \centering
            \includegraphics[width=7cm]{standard_coupling.pdf}
            \caption{結合強度図}
        \end{figure}

        既に述べたように今回作成した結合回路において強度をドメスティックに変えるパラメータはrf-SQUIDのスクリーニングパラメータとrf-SQUIDと共振器間の相互インダクタンスである。それぞれのパラメータを変えた時にどのように結合強度が変化するのを示した図が以下である。
        \begin{figure}[H]
            \centering
            \includegraphics[width=12cm]{standard_coupling_beta.pdf}
            \caption{結合強度の$\beta$依存性}
        \end{figure}
        また、外部磁束を0.5に固定子、$\beta$の値のみを変更することによる結合強度の対応をプロットすると$\beta$が0.8を超えたあたりで急激に結合強度が変化していることがわかる。
        \begin{figure}[H]
            \centering
            \includegraphics[width=8.5cm]{standard_coupling_betasweep.pdf}
            \caption{結合強度の$\beta$依存性}
        \end{figure}
        スクリーニングパラメータが1を超えるとこの関数は発散してしまうため、最適な動作点としては$0.8<\beta<0.95$付近が妥当であると考えられる。しかしながら、後述するが実際にはスクリーニングパラメータをこの領域内に収めることは非常に困難である。
        スクリーニングパラメータはジョセフソンインダクタンス$Lj$とrf-SQUIDのループインダクタンスLsにより$\beta=Ls/Lj$と表現されるが仮にLsの値を0.224[nH]で設計した場合
        \begin{equation*}
            L_{jdes} = 0.258 \pm 0.022\ [nH]
        \end{equation*}
        経験的にジョセフソン接合の作成にはインダクタンスにしてOOnHのばらつきが出るため、再現性が非常に低くなる。
        そこで、今回作成したサンプルには単一ジョセフソン接合の代替にdc-SQUIDを用いた。
        \subsection{$\beta$ ループによる変調}
        dc-SQUIDはジョセフソン接合2つを含んだ超伝導ループであり、この超伝導線路のインダクタンスは実効的に
        \begin{equation*}
            L_{jsq} = \frac{\Phi_0}{4\pi I_c|\cos(\pi(\frac{\Phi_{ext}}{\Phi_0}))|}
        \end{equation*}
        である。
        \begin{figure}[H]
            \centering
            \includegraphics[width=8.5cm]{dc-squid.pdf}
            \caption{dc-SQUIDによる$\beta$変調}
        \end{figure}
        この素子によりrf-SQUIDのジョセフソンインダクタンスを外部磁束によって変調することが可能となった。
        次にrf-SQUIDと共振器間の相互インダクタンスを向上させる手法について考える。
        今回採用した方法はミアンダインダクタンスを用いたものである。ミアンダインダクタンスは細線を蛇行させることにより各線路の相互インダクタンス、単純な線路長の増加によりインダクタンスを線路エリアに対して大きくすることができる。この設計において相互インダクタンスに注目することとなった経緯を説明する。
        修士の研究において大別して3種類のデバイスを測定したと述べたがこのうちJLCCの結果を受けてである。このサンプルはジョセフソン接合をrf-SQUIDと共振器間に挿入することで、ジョセフソンインダクタンスを用いて相互インダクタンスを強める目的で導入した。結果として望むような成果は得られなかったが次の2つの収穫が得られた。

        ①正方向の結合強度を向上する手段として相互インダクタンスの寄与は非常に大きいこと。
        
        ②相互インダクタンスはその大きさの2乗によって結合強度を増強すること。
        
        スクリーニングパラメータ$\beta$の精密な操作が結合強度を急激に増加させることは既に述べたが、結合素子として扱う際には磁束の急激な変化は望ましくない。操作が困難になることはもちろん急激な値の変化は解析をするさいにも困難を要する。そこで、まずは相互インダクタンスで可能な限りな結合強度の増強を試みる。
        また、共振器間の1次結合(直接的な相互インダクタンスを強めるためにrf-SQUIDの構造も縦長なループ構造へと修正した。
        \begin{figure}[H]
            \centering
            \includegraphics[width=12cm]{samplefigure.pdf}
            \caption{aMLCCの模式図}
        \end{figure}
    \subsection{メインループによる変調}
    

    \subsection{回路デザイン}

    \subsection{電磁界シミュレーション}
    

\section{製造}

    \subsection{製造パラメータ}

    \subsection{二重角度蒸着}

\section{測定サンプル}

    \subsection{回路パラメータ}
