\section{設計}
        今回測定したサンプルは3つ(1)rf-SQUID(2)rf-SQUIDwith twoJJ(3)aMLCCである。
        (1)は最も単純な構造であり、このサンプルが基本系となる。(2)は(1)の共振器と結合素子の接合部にジョセフソン接合を導入したものである。こうすることでガルバニック結合した結合素子ー共振器間の相互インダクタンスが増強される。この設計のポイントはこの相互インダクタンスの増強にある。結合強度を支配的に決定するパラメータは大きく2つスクリーニングパラメータと相互インダクタンスであるがこのうち相互インダクタンスを大きくするというのが狙いである。また局所的にインダクタンスを大きくすることで共振器の電流が結合部に集中し、磁気的結合は強くなる。最後に(3)は(2)のジョセフソン接合部をミアンダインダクタンスに代替し、rf-SQUIDのスクリーニングパラメータを決定するジョセフソン接合をdc-SQUIDに代替することで変調可能にしている。このサンプルが目的にしているのは第一にスクリーニングパラメータのオンデマンドな変調を可能にすること。第二に(2)のガルバニック接合部のジョセフソン接合を単純な磁気インダクタンスに代替することで物理モデルを単純化することである。ジョセフソン接合を導入することは大きなインダクタンスを得るためのもっとも簡単な方法である一方でデバイスに与える非線形な作用も大きくなることが予想される。結合素子として駆動する際単純な構造であること製造上も好ましい。

        では最も単純なタイプのサンプル(1)を例にしてサンプルの設計手順を説明する。
    \subsection{遷移スペクトル計算}
        まずは実験からどのように結合強度を見積もるのかということを説明する。既に今回作成したサンプルのハミルトニアンは示した。このハミルトニアンはSQUIDループを貫く磁束量子数に依存していることがわかる。各磁束量子数に対してハミルトニアンの対角化を行えば系の遷移周波数を知ることができる。
        この結果は解析的に見積もることができる。計算過程はやや煩雑であるので補足に記載したので参照されたい。
        計算結果からわかるように結合モードの遷移周波数間のエネルギー差はそのまま基底変換する前の結合強度を示している。よって実験から結合強度を見積もるにはこのスペクトルを設計値を初期条件に設定した上で関数フィッティングを行う。得られたパラメータから結合強度を見積もる。
        設計の段階でもこのことを念頭に置いた上で設計する。
        各変数を順に固定した後で測定環境を含め最適な値を探した。
        それぞれのサンプルの測定時期であるが
    \subsection{メインループによる変調}
    \subsection{$\beta$ ループによる変調}

    \subsection{回路デザイン}

    \subsection{電磁界シミュレーション}
    

\section{製造}

    \subsection{製造パラメータ}

    \subsection{二重角度蒸着}

\section{測定サンプル}

    \subsection{回路パラメータ}
