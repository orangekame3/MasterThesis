\section*{序章}
超伝導体を用いた擬似人工原子システムの確立\cite*{nakamura1999coherent}から約23年を経た現在、超伝導量子計算機の開発に注目が集まっている。量子計算機が古典コンピュータと比較して有意な結果を発揮するためには約100万もの擬似人口原子(以下量子ビット)を集積してからだといわれており、回路の集積度は年々増加している。
回路に装填される素子は量子ビットだけではない。量子情報を取得する読み出し素子。量子ビット同士を相互作用させる結合素子。量子情報を伝送させる伝送ライン。これら素子群をパッケージングする回路デザインも量子ビットの集積度に大きく寄与する。当研究室でも万能型量子計算機、専門特化型量子計算機の回路アーキテクチャ\cite*{Mukai2019}をそれぞれ提案しており、これらの提案は2次元平面状で実装可能であることが利点である。このアーキテクチャの特徴的な点は量子ビット同士を結合素子を介して直接結合させるのではなく、共振器を介していることである。共振器に準集中定数型の共振器を用いることにより回路デザインの拡張性も高い。その一方で量子ビット間に新たな素子が介入したことにより、共振器間の結合強度はかなり高強度にする必要がある。

本研究のテーマは上記前提のもと、共振器間の高強度可変結合素子の研究開発である。共振器間の結合素子には従来型のrf-SQUIDを拡張させたものを使用している。エンジニアリングな内容が主になるため本稿もそれに合わせた構成となっている。第1章で共振器結合素子開発の目的。第2章で前半で結合回路の物理モデルと使用する超伝導素子について記述する。第3章では実際に素子を作成する際にとった手順をパラメータの決定の仕方から電磁界シミュレーションまでを設計のパートで、製造パートではジョセフソン接合を製造する2重蒸着法について説明する。また章の最後では測定したサンプルの物理パラメータをまとめている。第4章では測定の手法と解析の手法を、第5章では測定結果を示す。第6章では前章の結果を受けた結合強度の解析を前半部で、後半部では回路を結合連成振動子とした物理のもとで解析を行う。第7章では結果を踏まえた総括を、第8章では今後の展望について記述する。

第10章では本文に記載するには冗長であるが重要である計算について補足という形でまとめてあるので参照されたい。