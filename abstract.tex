\section*{序章}
超伝導体を用いた擬似人工原子システムの確立\cite*{nakamura1999coherent}から約23年を経た現在、超伝導量子計算機の開発に注目が集まっている。量子計算機が古典コンピュータと比較して有意な結果を発揮するためには約100万もの擬似人口原子(以下量子ビット)を集積してからだといわれており、回路の集積度は年々増加している。
回路に装填される素子は量子ビットだけではない。量子情報を取得する読み出し素子。量子ビット同士を相互作用させる結合素子。量子情報を伝送させる伝送ライン。これら素子群をパッケージングする回路デザインも量子ビットの集積度に大きく寄与する。当研究室でも万能型量子計算機、専門特化型量子計算機の回路アーキテクチャをそれぞれ提案しており、2次元平面状で実装可能であることが利点である。