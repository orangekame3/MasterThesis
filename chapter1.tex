\section{アニーリング量子計算sss}
基底状態探索として近年量子アニーリングという手法がよく使われている。この計算手法は駆動ハミルトニアンをVと定義したときに系のハミルトニアンを
\begin{equation}
    \hat{H}(t) = s(t) \hat{H} +(1-s(t))\hat{V}
\end{equation}
と定義する。初期状態は$t=0$で$s(t)=0$、$t=t_f$で$s(t_f)=1$であることに注意する。
量子力学の断熱定理によれば系の初期状態を初期ハミルトニアンVで用意すれば逐次的にハミルトニアン
を変動させることで$t=t_f$ではハミルトニアン$H_f$において基底状態を実現することができるというものである。

上式によればエネルギーgapが小さければ小さいほどに演算に時間がかかるという結果が得られる。また実験で