\documentclass[uplatex,openany,oneside,a4j,11pt]{jsbook}
\usepackage[top=25truemm,bottom=25truemm,left=25truemm,right=25truemm]{geometry}
\usepackage[dvipdfmx]{graphicx}
\graphicspath{{./img/}}
\usepackage[dvipdfmx]{color}
\usepackage[dvipdfmx,colorlinks,linkcolor=blue,urlcolor=blue,bookmarks=true,bookmarksnumbered=true]{hyperref}
\usepackage{pxjahyper}
\usepackage{bookmark}
\usepackage{url}


%\usepackage[]{biblatex}
%\addbibresource{bibs}
\usepackage{booktabs}
\usepackage{amsmath}
\usepackage{amssymb}
\usepackage{amsfonts}
\usepackage{here}
\usepackage{comment}
\usepackage{subfigure}

%\usepackage{mediabb}
%\addbibresource{'C:/Users/miyao/Desktop/MasterThesis/bibs'}
%\usepackage{physics}
\usepackage[style=phys,articletitle=true,biblabel=brackets,chaptertitle=false,pageranges=false,backend=biber]{biblatex}
\addbibresource{bibs.bib}
% \usepackage{amsmath}
% \usepackage{amssymb}
% \usepackage{amsfonts}
% \usepackage{here}
% \usepackage{comment}
%\usepackage{physics}
%\usepackage[backend=biber]{biblatex}
%\usepackage[style=phys]{biblatex}
%\addbibresource{bibs}
\begin{document}
\begin{titlepage}
    \begin{center}
        {\Large 2020 年度 修士論文}\\
        \vspace{180truept}
        {\Huge 超伝導準集中定数共振器間における\\
        \vspace{10truept}
        高強度可変結合素子の研究開発}\\ 
        \vspace{70truept}

        {\Large \today}\\

        \vspace{70truept}

        {\Large 東京理科大学大学院 理学研究科物理学専攻 蔡研究室\\
        (学籍番号 1219537)}\\

        \vspace{20truept}

        {\huge 宮永 崇史}\\

        \vspace{160truept}
        {\Large 東京理科大学大学院 理学研究科物理学専攻}\\
    \end{center}
\end{titlepage}
%\maketitle
%\frontmatter
\addcontentsline{toc}{chapter}{序章}

\section*{序章}
超伝導体を用いた擬似人工原子システムの確立\cite*{nakamura1999coherent}から約23年を経た現在、超伝導量子計算機の開発に注目が集まっている。量子計算機が古典コンピュータと比較して有意な結果を発揮するためには約100万もの擬似人口原子(以下量子ビット)を集積してからだといわれており、回路の集積度は年々増加している。
回路に装填される素子は量子ビットだけではない。量子情報を取得する読み出し素子。量子ビット同士を相互作用させる結合素子。量子情報を伝送させる伝送ライン。これら素子群をパッケージングする回路デザインも量子ビットの集積度に大きく寄与する。当研究室でも万能型量子計算機、専門特化型量子計算機の回路アーキテクチャをそれぞれ提案しており、2次元平面状で実装可能であることが利点である。
\setcounter{tocdepth}{2}
\tableofcontents
%\mainmatter

\chapter{目的}
    \begin{abstract}
        研究の目的
    \end{abstract}
    \section{超伝導回路の大規模化}
    本研究の目的は共振器間の高強度結合素子の開発である。まずはこの研究のモチベーションについて説明することから始める。
    序章でも述べたが超伝導量子回路の集積度は年々向上しており、知名度の高い企業が研究開発に乗り出していることからも世間からの注目度は非常に高い。しかしながら、各々の量子ビットを効率よく相互作用させようとする際にはまだまだ課題も多い。超伝導量子回路を用いた大規模な集積回路として有名なのはD-wave社の量子アニーリング回路、Googleの万能型量子回路であるがその2つの例を見ても各々の量子ビットの全結合は実現できていない。量子ビットと結合素子を直接結合させるには回路デザインの観点から限界が生じている。我々の研究チームでは量子アニーリング型及び万能型のそれぞれについて2次元回路アーキテクチャの開発に取り組んでいる。この2つのアーキテクチャのうち特に量子アニーリング回路では本稿のテーマである。共振器間の高強度結合素子の開発が要となる。以下に大規模アニーリング回路を駆動するために必要なここの超伝導素子のパラメータを示す。

    この図には量子ビット間の実質的な結合強度Jijと量子ビットのパラメータが示されている。量子アニーリングはイジンぐモデルを前提に構築されたモデルであり、最終的な解は結合強度Jijにマッピングされる。始状態のパラメータは量子ビットの遷移周波数にマップされるため始状態と終状態において、この2つのパラメータはバランスがとれていることが前提となる。またスイープ時間において突発的な状態遷移を避けるために量子ビットの遷移周波数はGHz帯に設定する必要がある。この状況においてJijの強度をGHz体に保つためには共振器間の結合強度の絶対値には少なくとも400MHzが必要とされる。マッピングの自由度を上げるにはこのJijは正負でのバランスがとれていることが望ましい。
    以上より共振器間の結合に要請される理想的なパラメータは-400MHzから400MHzを自由に変調できるものである。
    先行研究\cite*{Wulschner2016}では-320MHzから37MHzの結合強度を実現しているが、正負両方向において強度不足であることがわかる。

\section{}

\section{結合振動子}
    本研究のテーマはエンジニアリングな側面が強いが、研究対象としている結合共振回路は物理モデルとしても非常に興味深い内容である。量子力学と古典力学のアナロジーとして連成振動子モデルは多くの研究がなされている。\cite*{Rodriguez2016}\cite*{Ivakhnenko2018}\cite*{Novotny2010}
    上記のモデルは一般的な共振素子(振動子)と共振素子を結合した系を理論的視点から論じている。今回作成した共振回路は共振器間のダイレクトな結合強度を変調できるという点で非常に魅力的である。一般的な共振素子はその遷移周波数がすべて同一であり、共振回路の周波数を持った光子を外部から入力しても高準位へと遷移してしまい、光子の放出現象は示さない。しかし、振動子間に結合項が存在すると振る舞いは一変し新たな固有モードが出現する。古典力学ではしばしばこのモードのことを基準モードと呼んでいる。この
    \begin{figure}[H]
        \centering
        \includegraphics[width=9cm]{b.pdf}
        \caption{エネルギースペクトル}
    \end{figure}
    \begin{figure}[H]
        \centering
        \includegraphics[width=9cm]{a.png}
        \caption{エネルギースペクトル}
    \end{figure}
    \begin{figure}[H]
        \centering
        \includegraphics[width=9cm]{c.pdf}
        \caption{エネルギースペクトル}
    \end{figure}




\chapter{原理}
    \begin{abstract}
        超伝導回路の紹介と結合素子の物理モデルの導入
    \end{abstract}
    \section{物理モデル}
    本章ではまず結合共振回路の物理モデルの説明から始める。
\section{超伝導回路素子}
    サンプル作成に使用した超伝導回路素子の説明を包括的に行う。まずは超伝導共振器の導入を行う。
    \subsection{超伝導共振器}
        超伝導量子エレクトロニクスの分野ではしばしば共振回路が使用される。目的は様々であるが、量子ビットを外部から遮断するためのフィルターとしてや量子ビット同士を結合させるバスとしても利用される。
        \subsubsection{超伝導分布定数回路}
            最も使用頻度の高いデザインは平面型共振回路(CPW)である。CPW型共振回路の特徴は伝送ラインの断面図が任意の点で50Ωに保たれていることである。グランドラインと伝送ラインの間の溝を常に一定に保つことで共振回路長を調節するだけで共振周波数を調節することができる。この時共振回路全体のインダクタンスとキャパシタンスは単位あたりのインダクタンスLp.u.lとキャパシタンスCp.u.lを共振回路長で乗算したものである。このように回路の共振パラメータが一様に分布したモデルで記述される共振器のことを分布定数型共振回路と呼ぶ。またインピーダンス不整合が抑制されているため共振回路のQ値も特別な処理を施さずとも高い。CPW型の共振周波数がどのような物理モデルによって記述されるのかは補足に記したので参照されたい。
        \subsubsection{超伝導準集中定数回路}
            分布定数型の共振回路のデメリットはデザインに自由度がないことである。多くの超伝導素子を1つのサンプルに集積する際には回路デザインは非常に重要である。量子ビットのQ値は高く保って置きたいため、次にデザインに自由度を持たせるとすれば共振器になる。そういったケースを考えるとCPWのような共振回路では回路作成に困難を要する。
            また、共振回路同士の結合を考える場合、その結合方式は電気的結合(キャパシティブ結合)か磁気的結合(インダクティブ結合)になる。高強度の結合を考える際、結合にはどちらか一方のみを考える方が望ましい。両方の結合項を同時に記述すれば結合強度に対してそれぞれ逆符号をとっていることがわかる。今回作成したサンプルではどれもインダクティブな結合方式を利用することを考えた。
            この場合結合には電磁誘導の法則が適用されるため、結合強度を決めるパラメータは共振器間の実質的な相互インダクタンスと共振器に流れる電流に依存する。つまり共振器に流れる電流は大きい方がよい。共振回路を流れる電流は共振回路のインダクタンスを用いて以下のように記述される。つまり同一の周波数を持つ共振器において、キャパシタンス部分は小さく、インダクタンス部分が大きいことが共振器としては望ましいことがわかる。こういった調整をCPW型の共振器で行うことは難しい。局所的なパラメータを調整する際には今回採用した集中定数素子型の共振器wお使用することが理想的であるといえる。
            分布定数回路に対して共振素子のインダクタンスとキャパシタンスが部分的に集中した回路のことを集中定数素子型の共振器と呼ぶ。この場合、CPW型の共振素子とはその特性が異なる。CPW型の共振器はキャパシタンスを開放端、グランドを固定端とした空洞間に閉じ込められた音響モードと物理的には全く同じ振る舞いをする。基本モードの整数倍の共振周波数で共鳴し、両端開放型の共振器を$\lambda/2$型の共振器、開放ー固定型の共振器を$\lambda/4$型共振器と呼び、通常はそれぞれの基本モードのみを使う。
    \subsection{ジョセフソン接合}
            ジョセフソン接合は超伝導量子エレクトロニクスという研究分野の根幹をなす素子である。超伝導量子回路を構成するあらゆる素子にジョセフソン接合は使われている。使用用途は様々であるが例えば量子ビットを作成することを考える際には共振回路に非調和性をもたらす素子として導入される。前小節で述べたように共振回路の遷移周波数はすべて同一である。同一の周波数を持っていることは例えば量子ビットを作ろうと考えた時には、適していないことがわかる。量子ビットの状態をその遷移周波数で励起しても高準位の遷移周波数も同一であるためいつまでたっても励起し続けてしまうためである。
            本研究では結合素子であるrf-SQUIDを構成する素子としてジョセフソン接合を導入する。

            多方面である使用されるジョセフソン接合であるがその基本的な物理現象についてこの小節にて説明する。
            2つの超伝導体を薄い常伝導金属、半導体などを介してサンドウィッチするような接合のことを、josephson接合と呼ぶ\\
            この接合により生じる物理的現象を利用して我々は量子ビットやSQUID、その他微細デバイスを作成している。\\
            ここではjosephson接合による物理的効果を理論から解説していくこととする。

    
            この接合により生じる物理的現象を、発見者の名に因んでjosephson効果と呼ぶ。
            この現象は大別するとACjosephson効果、DCjosephson効果の2つに分けることができる。
            この2つの効果についてGL理論から解説を始めることとする。
            \subsubsection{dc josephson効果}
                josephsonが1962年に行った理論的予言によれば、2つの超伝導体の間にゼロ電圧下で以下のような超伝導電流が流れるとしている。
                \begin{equation*}
                    I_s = I_c \sin(\Delta \psi)
                \end{equation*}
                ここで$\Delta \psi$とはGL波動関数の位相差である。また、臨界電流$I_c$は接合に流すことのできる、最大の超伝導電流である。
                彼はさらに接合に電位差vが生じているときに位相差が次のように振動すると予言した。
                \begin{equation*}
                    \frac{d(\Delta \psi)}{dt} = \frac{2eV}{\hbar}
                \end{equation*}
                これにより、電流は振幅を臨界電流$I_c$、周波数を$\nu = \frac{2eV}{h}$の交流となる。
                つまり、この電流変化はエネルギー$h\nu$でクーパー対が接合を通過するエネルギーと一致していることがわかる。
                以上2つの関係式によりこの接合により蓄えられるエネルギーは
                \begin{eqnarray}
                    \int_0^{t} (I_s V)dt&=&\int_{0}^{\delta} I_s(\hbar/2e)d\delta\\
                    &=&E_j(1-\cos(\delta))\\
                \end{eqnarray}
                と表現することができる。ここで$E_j=\hbar I_c/2e$である。
    \subsection{rf-SQUID}
                
    \subsection{dc-SQUID}

    \subsection{カイネティックインダクタンス}
    
    \subsection{ミアンダインダクタンス}


\chapter{実装}
    \begin{abstract}
        結合素子の設計手法と作成方法を紹介。測定に使用したサンプルについても言及
    \end{abstract}
    \section{設計概要}
\subsection{初期サンプル}
前章では最もベーシックなrf-SQUID型の結合回路について解説した。本研究の初期段階においても先行研究に倣い単純なrf-SQUID型の結合回路を作成することから始めた。
\begin{figure}[H]
    \centering
    \includegraphics[width=12cm]{theory_circuit1_1.pdf}
    \caption{LEAl}
\end{figure}
作成したサンプルはそれぞれCPW型の共振器、LE型共振器の2つを採用した。また、CPW型共振器には超伝導ループとしてNb(ジョセフソン接合はAl)をLE型についてはNbで作成したものAlで作成したものの2パターンを作成した。それぞれCPWNb,LENb,LEAlとラベリングする。\\
測定結果については測定の章で詳細に記述するが、この3タイプのサンプルのうち有意な結果が得られたのはLEAlであった。よって本章ではLEAlをベースに結合性能の向上にむけて採用したアプローチを中心に解説する。\\
\subsection{中期サンプル}
前章で解説したように結合強度を決定するのはrf-SQUIDを介した実効相互インダクタンス$M_{eff}$と共振器間の磁気的1次結合$M_{ab}$の和である。このうち$M_{ab}$は距離に強く依存する。
\begin{equation}
    M_{eff}(\Phi)  = \frac{L_0^2}{L_{rf}(\Phi)}=\frac{L_0^2}{L_{S}} \frac{\beta \cos \left(2 \pi \Phi/\phi_{0}\right)}{1+\beta \cos \left(2 \pi \Phi/\phi_{0}\right)}
\end{equation}
$M_{eff}$の強度増強を試みる場合考えられる方法は大きく2つ
\begin{enumerate}
    \item rf-SQUID間と共振器の相互インダクタンス$L_0$の強度を増強する
    \item rf-SQUIDのスクリーニングパラメータ$\beta$を限りなく1に近づける
 \end{enumerate}
このうち、まずはrf-SQUID間と共振器の相互インダクタンス$L_0$の強度を増強することを第1に考えた。理由はスクリーニングパラメータを$\beta<1$を満たしつつ限りなく1に漸近させることが設計上困難なためである。$\beta$が1を超えてしまうとrf-SQUIDを貫く磁束は外部磁束に対して2価の関数になると同時にポテンシャルの安定点はループを貫く磁束に対してダブルポテンシャルをとるようになる。共振器間の安定な結合を得るには$\beta<1$は必ず満たさなければならない条件である。
\begin{figure}[H]
    \begin{minipage}[t]{0.5\columnwidth}
        \centering
        \includegraphics[clip, width=1.0\columnwidth]{rfsquid_current.pdf}
        \caption{外部磁束に対する実効磁束の挙動の変化}
    \end{minipage}%
    \begin{minipage}[t]{0.5\columnwidth}
        \centering
        \includegraphics[clip, width=1.0\columnwidth]{rfsquid_potential.pdf}
        \caption{実効磁束に対するポテンシャルエネルギーの挙動の変化}
    \end{minipage}
\end{figure}
本研究においてはrf-SQUIDと共振器間の自己インダクタンス$L_0$を強めるためにガルバニック結合を採用している。そこでさらなる自己インダクタンスの増強に向けてジョセフソン接合をガルバニック結合に付与することを考えた。ガルバニック結合にジョセフソン接合を採用する事例は共振器と磁束量子ビットの深強結合等の先行研究\cite*{Yoshihara2017}があり、これに倣った形である。先行研究ではさらにジョセフソン接合をdc-SQUIDに置き換えることで強力な結合強度を得ることに成功している。
本研究では自己インダクタンス$L_0$をジョセフソンインダクタンス$L_J$で置き換える形で回路を作成した。このサンプルを以下LEJとラベリングする。LEJの設計は伊藤氏、鈴木氏の仕事である。
\begin{figure}[H]
    \centering
    \includegraphics[width=12cm]{theory_circuit2_1.pdf}
    \caption{LEJ}
\end{figure}
\subsection{後期サンプル}
サンプルの詳細な結果については次章で解説するが結果のみ述べればLEAlと比較して正方向に於ける結合強度が増強し、負方向に於ける結合強度は減少した。
この結果についてそれぞれ考えられる理由としては
\begin{enumerate}
    \item $L_J$の追加によって$L_0$が実効的に増加した
    \item ループサイズがLEAlよりも小さいこと
    \item rf-SQUIDのスクリーニングパラメータのが設計通りにならなかったこと
 \end{enumerate}
 等が考えられる。\\
 ジョセフソン接合を追加することによって自己インダクタンスを増強することができた一方で、測定したスペクトルにはLEAlでは観測されなかった非線形な作用が生じた。これは結合素子として扱うには回路自体を複雑にしてしまうため、望ましくない。また、rf-SQUID-共振器間の自己インダクタンス$L_0$がいくら大きくとも、rf-SQUIDのスクリーニングパラメータ$\beta$が適切な値にならなければ高強度な結合は得られない。逆に、緻密にスクリーニングパラメータを決定することができれば非常に高強度な結合を得ることができる。
 \begin{figure}[H]
    \begin{minipage}[t]{0.5\columnwidth}
        \centering
        \includegraphics[clip, width=1.0\columnwidth]{standard_coupling_beta.pdf}
        \caption{結合強度の$\beta$依存性}
    \end{minipage}%
    \begin{minipage}[t]{0.5\columnwidth}
        \centering
        \includegraphics[clip, width=1.0\columnwidth]{standard_coupling_betasweep.pdf}
        \caption{外部磁束を最適動作点に固定した状態での$\beta$依存性}
    \end{minipage}
\end{figure}
 ここまでの結果から考慮すべき点をまとめると
 \begin{enumerate}
    \item ジョセフソン接合を用いずに高強度な自己インダクタンス$L_0$を達成すること
    \item ループの形状は共振器に対して縦長な構造をとることで、共振器間距離を狭めること
    \item rf-SQDUIのスクリーニングパラメータ$\beta$を適切に製作できるような設計にすること
 \end{enumerate}
本研究において上記の条件を達成するためにとった手法は
\begin{enumerate}
    \item rf-SQUID-共振器間の自己インダクタンスにミアンダインダクタンスを導入
    \item 測定時にスクリーニングパラメータ$\beta$を決定できるようにdc-SQUIDを導入
 \end{enumerate}
 である。このミアンダインダクタンスを追加したサンプルを以下LEMとラベリングする。
 \begin{figure}[H]
    \centering
    \includegraphics[width=12cm]{theory_circuit3_1.pdf}
    \caption{LEM}
\end{figure}
ミアンダインダクタンスについては既に前章で解説したため、ここではジョセフソン接合の代わりにrf-SQUIDを用いることでスクリーニングパラメータの変調が可能となることを以下に示す。\\
LEMの回路図中結合素子部分のみに注目した図について
    %     今回測定したサンプルは3つである。作成時期順に列挙すると、基本形となるrf-SQUIDを共振器間に配置したLCC。そしてrf-SQUIDと共振器間の結合部にジョセフソン接合を導入したJLCC、最後に接合部をミアンダインダクタンスに変更したMLCCである。まずは基本型の構造は先行論文がとっている手法と全く同じである。
    %     \begin{figure}[H]
    %         \centering
    %         \includegraphics[width=14cm]{sample.pdf}
    %         \caption{サンプル図}
    %     \end{figure}
    %     上図は作成サンプルの回路図を模式的に表したものである。共振器とrf-SQUID間は線路に流れる電流により生じる磁場を介して結合している。模式図のように素子間が完全に接地した状態の結合のことをGalvanic結合と呼ぶ。この構造では、素子間の相互インダクタンスが接地部分の自己インダクタンスとほぼ等しくなるため非常に強力な相互インダクタンスを得ることができる。

    %     設計のポイントは①rf-SQUID-共振器間の相互インダクタンス増強と②rf-SQUIDスクリーニングパラメータの精密な製造である。この2つのパラメータは結合強度を向上する上で非常に重要な因子となる。
    %     この2つのパラメータが結合強度向上に大きく寄与するということを基本型であるrf-SQUIDの遷移スペクトルの計算を通して論拠する。
    % \subsection{遷移スペクトル計算}
    %     基本型の回路のハミルトニアンは
    %     \begin{equation}
    %         \hat{\mathcal{H}}=\hbar\left(\hat{a}^{\dagger }\ \hat{b}^{\dagger }\right)\left(\begin{array}{cc}
    %         \tilde{\omega}_{a} & g(\Phi ) \\
    %         g(\Phi ) & \hat{\omega}_{b}
    %         \end{array}\right)\left(\begin{array}{l}
    %         \hat{a} \\
    %         \hat{b}
    %         \end{array}\right)
    %     \end{equation}
    %     \begin{equation}
    %         = \hbar \hat{\omega}_{a} \hat{a}^{\dagger} \hat{a}+\hbar \hat{\omega}_{b} \hat{b}^{\dagger} \hat{b}+\hbar g(\Phi)\left(\hat{a}^{\dagger}\hat{b}+\hat{a} \vec{b}^{+}\right)
    %     \end{equation}

    %     である。式の前項2つが左右各共振器の調和振動子ポテンシャルである。最終項が結合項であり、この結合項に依存して2つの共振器が反発することを示す。各共振器の生成消滅演算子を
    %     \begin{equation}
    %         \hat{c}_{\pm}=\frac{\hat{a} \pm \hat{b}}{\sqrt{2}} \quad \hat{c}_{+}^{\dagger}=\frac{\hat{a}^{\dagger} \pm \hat{b}^{\dagger}}{\sqrt{2}}
    %     \end{equation}

    %     と置き換える。ここで演算子$\hat{c}_{\pm}$は各共振器の状態が混合した状態である。このように変換を行うと最終的に得られる状態は
    %     \begin{equation}
    %         \hat{H}=\hbar \Omega_+\hat{c}^{\dagger}_+\hat{c}_+ + \hbar \Omega_-\hat{c}^{\dagger}_-\hat{c}_- + \hbar \Delta\left(\hat{c}^{\dagger}_+ \hat{c}_- +\hat{c}^{\dagger}_- \hat{c}_{+}\right)
    %     \end{equation}

    %     \begin{equation}
    %         =\hbar\left(\begin{array}{cc}
    %         \hat{c}^{\dagger}_{+} & \hat{c}^{\dagger}_-
    %         \end{array}\right)\left(\begin{array}{cc}
    %         \Omega_{+} & \Delta \\
    %         \Delta & \Omega_{-}
    %         \end{array}\right)\left(\begin{array}{l}
    %         \hat{c}_{+} \\
    %         \hat{c}_{-}
    %         \end{array}\right)
    %     \end{equation}

    %     となる。この時$\Omega_+ = (\omega_a+\omega_b)/2 + g$、$\Omega_- = (\omega_a+\omega_b)/2 - g$、$\Delta = (\omega_a-\omega_b)/2$である。これより、2つの独立な調和振動子系は結合項$g$によって新たな基準モードで書き表すことができる。この新たな基底同士は元々の共振周波数の離調$(\omega_a-\omega_b)/2$で書き表すことができる。また変換前の基底、つまり左右の調和振動系の結合項$g$は新基底において新固有値の差分に現れる。これを図示すると以下のようになる。
    %     \begin{figure}[H]
    %         \centering
    %         \includegraphics[width=16cm]{standard_eigen.pdf}
    %         \caption{遷移周波数図}
    %     \end{figure}
    %     つまり、測定において得られた2つの基準モードの差分を計算することにより元のハミルトニアンの結合強度gを見積もることができる。以下に上図に対応する外部磁束に対応する結合強度の図をプロットする。
    %     \begin{figure}[H]
    %         \centering
    %         \includegraphics[width=7cm]{standard_coupling.pdf}
    %         \caption{結合強度図}
    %     \end{figure}

        % 既に述べたように今回作成した結合回路において強度をドメスティックに変えるパラメータはrf-SQUIDのスクリーニングパラメータとrf-SQUIDと共振器間の相互インダクタンスである。それぞれのパラメータを変えた時にどのように結合強度が変化するのを示した図が以下である。

        % また、外部磁束を0.5に固定子、$\beta$の値のみを変更することによる結合強度の対応をプロットすると$\beta$が0.8を超えたあたりで急激に結合強度が変化していることがわかる。
        % % \begin{figure}[H]
        % %     \centering
        % %     \includegraphics[width=8.5cm]{standard_coupling_betasweep.pdf}
        % %     \caption{結合強度の$\beta$依存性}
        % % \end{figure}
        % スクリーニングパラメータが1を超えるとこの関数は発散してしまうため、最適な動作点としては$0.8<\beta<0.95$付近が妥当であると考えられる。しかしながら、後述するが実際にはスクリーニングパラメータをこの領域内に収めることは非常に困難である。
        % スクリーニングパラメータはジョセフソンインダクタンス$Lj$とrf-SQUIDのループインダクタンスLsにより$\beta=Ls/Lj$と表現されるが仮にLsの値を0.224[nH]で設計した場合
        % \begin{equation*}
        %     L_{jdes} = 0.258 \pm 0.022\ [nH]
        % \end{equation*}
        % 経験的にジョセフソン接合の作成にはインダクタンスにしてOOnHのばらつきが出るため、再現性が非常に低くなる。

        \begin{figure}[H]
            \centering
            \includegraphics[width=12cm]{dc-squid_circuit.pdf}
            \caption{dc-SQUIDの回路図}
        \end{figure}
        dc-SQUIDを流れる電流$I_{tr}$は2つの電流$I_{c1},I_{c2}$により
        \begin{equation}
            I_{tr} = I_{c1} + I_{c2}
        \end{equation}
        と書き表せる。また、超伝導ループを周回する電流を$I_{cir}$とすると2つの電流は
        \begin{subequations}
            {\jot=10pt
        \begin{eqnarray}
            I_{c1}  = I_{tr}/2+I_{cir}\\
            I_{c2}  = I_{tr}/2-I_{cir}
        \end{eqnarray}
        }
        \end{subequations}
        である。ジョセフソン接合の位相差をそれぞれ$\phi_1,\phi_2$とすると臨界電流$I_c$において
        \begin{subequations}
            {\jot=10pt
        \begin{eqnarray}
            I_{c1}  = I_{c}sin(\phi_1)\\
            I_{c2}  = I_{c}sin(\phi_2)
        \end{eqnarray}
        }
        \end{subequations}
        とも表記できる。超伝導ループの周期条件より位相差$\phi_2-\phi_1$について$2\pi$の整数倍が要請され
        \begin{equation}
            \phi_{2}-\phi_{1}=\frac{2 \pi}{\Phi_{0}} \Phi_{\mathrm{dc}}=\frac{2 \pi}{\Phi_{0}}\left(\Phi_{\mathrm{ext}}+L_{dc} I_{cir}\right)
        \end{equation}
        ここで$\Phi_{dc}$はdc-SQUIDを貫く磁束の総和であり、外部磁場$\Phi_{ext}$とループが持つ自己インダクタンスによる遮蔽磁場$\L_{dc}I_{cir}$の和である。dc-SQUIDのスクリーニングパラメータが
        \begin{equation}
            \beta_{dc}:=\frac{2 L I_{c}}{\Phi_{0}} \ll 1
        \end{equation}
        であることを仮定すると、ループを周回する電流とジョセフソン接合の臨界電流は$I_{cir}<I_{c}$であるため、ループを貫く磁場は実質$\Phi_{dc} = \Phi_{ext}$であると考えてよい。すなわち
        \begin{equation}
            I=I_{c}\left[\sin \phi_{1}+\sin \left(2 \pi \frac{\Phi_{\text {ext }}}{\Phi_{0}}+\phi_{1}\right)\right]
        \end{equation}
        ここで
        \begin{equation}
            \chi:=\phi_{1}+\pi \frac{\Phi_{\text {ext }}}{\Phi_{0}}
        \end{equation}
        と置くことで三角関数の加法定理より
        \begin{equation}
            I=2 I_{c} \cdot \sin \chi \cdot \cos \left(\pi \frac{\Phi_{\mathrm{ext}}}{\Phi_{0}}\right)
            \end{equation}
        が導かれる。$\sin\chi$は-1から1の範囲で変動するため、dc-SQUIDを流れる電流の最大値は
        \begin{equation}
            I_{s, \max }=2 I_{c}\left|\cos \left(\pi \frac{\Phi_{\text {ext }}}{\Phi_{0}}\right)\right|
        \end{equation}
        と求まる。よってdc-SQUIDは外部磁場によって変調可能なジョセフソンインダクタンス
        \begin{equation}
            L_{\mathrm{dcSQUID}}\left(\Phi_{\mathrm{ext}}\right)=\frac{\hbar}{2 e I_{s, \max }}=\frac{\Phi_{0}}{4 \pi I_{c}\left|\cos \left(\pi \frac{\Phi_{a}}{\Phi_{0}}\right)\right|}
        \end{equation}
        として扱うことが可能になる。
        rf-SQUID中の単一ジョセフソン接合をdc-SQUIDに置き換えることにより新たにスクリーニングパラメータ$\beta_{eff}$を定義し直すと
        \begin{equation}
            \beta_{eff}(\Phi_{ext}) = \frac{4\pi L_{s}I_{c}\left|\cos \left(\pi \frac{\Phi_{a}}{\Phi_{0}}\right)\right|}{\Phi_0}=\frac{2L_s\left|\cos \left(\pi \frac{\Phi_{a}}{\Phi_{0}}\right)\right|}{L_{J0}}
        \end{equation}
        以下に外部磁束に対するスクリーニングパラメータの応答を示す。
        \begin{figure}[H]
            \centering
            \includegraphics[width=8cm]{dc-squid.pdf}
            \caption{dc-SQUIDによる$\beta$変調}
        \end{figure}
        % この素子によりrf-SQUIDのジョセフソンインダクタンスを外部磁束によって変調することが可能となった。
        % 次にrf-SQUIDと共振器間の相互インダクタンスを向上させる手法について考える。
        % 今回採用した方法はミアンダインダクタンスを用いたものである。ミアンダインダクタンスは細線を蛇行させることにより各線路の相互インダクタンス、単純な線路長の増加によりインダクタンスを線路エリアに対して大きくすることができる。この設計において相互インダクタンスに注目することとなった経緯を説明する。
        % 修士の研究において大別して3種類のデバイスを測定したと述べたがこのうちJLCCの結果を受けてである。このサンプルはジョセフソン接合をrf-SQUIDと共振器間に挿入することで、ジョセフソンインダクタンスを用いて相互インダクタンスを強める目的で導入した。結果として望むような成果は得られなかったが次の2つの収穫が得られた。

        % ①正方向の結合強度を向上する手段として相互インダクタンスの寄与は非常に大きいこと。
        
        % ②相互インダクタンスはその大きさの2乗によって結合強度を増強すること。
        
        % スクリーニングパラメータ$\beta$の精密な操作が結合強度を急激に増加させることは既に述べたが、結合素子として扱う際には磁束の急激な変化は望ましくない。操作が困難になることはもちろん急激な値の変化は解析をするさいにも困難を要する。そこで、まずは相互インダクタンスで可能な限りな結合強度の増強を試みる。
        % また、共振器間の1次結合(直接的な相互インダクタンスを強めるためにrf-SQUIDの構造も縦長なループ構造へと修正した。
        % \begin{figure}[H]
        %     \centering
        %     \includegraphics[width=12cm]{samplefigure.pdf}
        %     \caption{aMLCCの模式図}
        % \end{figure}
        % 次に配線構造を考える。サンプル上に載せる配線の本数は少なければ少ないほどに良い。特に磁束バイアスを用いる場合、クロストークを考慮する必要がある。今回のサンプルでは少なくとも2つの独立な電流源が必要となる。1つはdc-SQUIDにバイアスしてスクリーニングパラメータを調整する電流源、もう一つがrf-SQUIDのメインループを貫き、結合素子の強度を変更するための磁束バイアスである。ここでは配線構造について考える。
        % 実験環境ではサンプル上に載せるオンチップバイアスラインとサンプルをマウントするサンプルホルダー上に積載しているグローバルフラックスを用いることができる。配線本数を減らすという観点ではこのグローバルフラックスを利用するのが好ましい。しかしながら、グローバルフラックスはサンプル全体に均一な磁場がかかるため、メインループとdc-SQUIDのループ面積比を極端に差別化することで独立な操作がしやすいように工夫した。
        % \begin{figure}[H]
        %     \centering
        %     \includegraphics[width=10cm]{配線.pdf}
        %     \caption{aMLCCの模式図}
        % \end{figure}
        % 一方をグローバルフラックスで駆動し、もう一方をオンチップバイアスで駆動することを考える場合、配線の仕方は上記図のような2つの方法が考えられる。今回はメインループ側にオンチップバイアスを取り付けた。グローバルバイアスを全体に印加しているためメインループ、dc-SQUIDを均一磁場が貫く。貫く磁束量子の本数はループの面積比に依存するため、メインループの面積比をdc-SQUIDの100倍にすることで差別化した。まずは上部の配線構造について説明を加えるとこの配線構造ではグローバルバイアスでrf-SQUIDを貫く磁束を固定した上で新たにオンチップバイアスでdc-SQUIDに磁束を印加することでrf-SQUIDのスクリーニングパラメータを調整することができる。この配線構造で問題となるのはオンチップバイアスラインとrf-SQUIDのメインループが非常に近接しているということである。クロストークを可能な限り抑制することを考えるとこの配線は望ましくない。
        % 次に下部の配線構造であるが、この場合ではグローババイアスでdc-SQUIDを貫く磁束を固定した上でrf-SQUIDを貫く磁束をオンチップバイアスで調節することを目的としている。この場合オンチップバイアスがdc-SQUIDに影響するクロストークは非常に小さいといえる。ループの形状を縦長にするという工夫はクロストークを抑制するという点でも非常に合理的であることがわかる。しかし、クロストークがいかに小さいとはいえ、オンチップバイアスからdc-SQUIDに寄与するクロストークは少なからず存在する。測定を行う際にはそれぞれの電流源がそれぞれのループに寄与するクロストークをインダクタンス行列を用いて評価する。

    \section{共振器の電磁界シミュレーション}
        共振器の電磁界シミュレーションについて解説する。前章ではCPW型共振器、LE型共振器についてそれぞれ解説したが、どちらのタイプも電磁界シミュレーターを適切に利用することで共振周波数を推定することができる。当研究室では回路の描画ソフトウェアとしてAutoDesk社のAutoCAD(ver.2018)を、電磁界シミュレーターはAWR社のマイクロウェーブオフィス(ver14.03)を使用している。\\
        ここでは後期サンプル(LEM)を例に解説するが、CPW型についても電磁界シミュレーターの扱いは同じである。シミュレーターを利用する前の回路描画の段階で$\lambda/2$長を算出する方法は前章で解説した通りコンフォーマルマッピングにのっとって解析的に計算、もしくはソフトウェアを用いる方法がある。ただし、本研究の設計においてCPW型はBroadcom社のAppCAD Design Assistantを利用した。このソフトウェアは回路に使われてている素材(Si基板、Nb薄膜)のパラメータを適切に入力することで求めたい周波数の1波長の長さを計算することができる。しかしながら、Microwave Office,AppCAD Design Assistantどちらのソフトウェアについてもカイネティックインダクタンスに依る寄与は含まれていないため注意が必要である。\\
        まずはAutoCADを用いてサンプルデザインを書き出す。
        \begin{figure}[H]
            \centering
            \includegraphics[width=14cm]{design3.pdf}
            \caption{LEM回路デザイン}
            \label{LEM}
        \end{figure}
        図\ref*{LEM}はAutoCADで作図したものをLayoutEditorを用いてファイル変換$\rightarrow$KLayoutで着色したものである。図中の青、赤部分が共振器に相当する。どちらの共振器も全く同じデザインであるため、それぞれ個別に測定した場合、理想的には同一の共振周波数が観測されるはずである。図\ref{LEM}のように同一共振モードの共振器を並べて配置すると、磁気的1次結合により共振周波数がシフトする。シフト幅は元の共振周波数からそれぞれ反発する方向に$g/2$である。この共振モードのシフトは電磁界シミュレーションでも確認することができる。以下単一共振器、結合共振器を別々にシミュレートすることで共振モード、及び1次結合の結合強度gを推定する。
        \begin{figure}[H]
            \begin{minipage}[t]{0.5\columnwidth}
                \centering
                \includegraphics[clip, width=1.0\columnwidth]{resonator_design_for_mwo.pdf}
                \caption{単一共振器の電磁界シミュレーション}
                \label{単一共振モード}
            \end{minipage}%
            \begin{minipage}[t]{0.5\columnwidth}
                \centering
                \includegraphics[clip, width=1.0\columnwidth]{coupled_resonator_mwo.pdf}
                \caption{結合共振器の電磁界シミュレーション}
                \label{結合共振モード}
            \end{minipage}
        \end{figure}
        AutoCADで書き出したファイルをMicrowave Office上に読み込み、インプット・アウトプットポート及びグランドを配置したものが図\ref*{単一共振モード}である。上図の例ではポート番号1がインプット、ポート番号2がアウトプット、ポート番号-1がグランドに相当する。ポートの配置により、異なる結果が帰ってくる場合があるので、幾度が試行して適切な位置を決定する。\\
        シミュレーターでは、諸所の物理量について計算を行えるが、今回はS11の反射測定を計算した。以下がその結果である。
        
        % 共振器には超伝導準集中定数素子を用いた。超伝導量子回路で一般的に使用されているCPW型共振器はグランドと伝送線路の距離を伝送損失の少ない50$\Omega$で保つことで単位長さあたりのキャパシタンス$Cp.u.l$とインダクタンス$Lp.u.l.$を求め、伝送線路の長さを乗じることで回路全体のキャパシタンスとインダクタンスを決定している。非常にこの構造は伝送損失が少なく非常に簡便に共振器の設計ができる点がメリットとなる。他方で単位長さあたりのキャパシタンスとインダクタンスが一定にするため複雑な構造をつ繰り出すことはできない。また集中定数回路のように局所的に共振パラメータを調節することには向かないといえる。他方で今回採用した準集中定数型の共振器は意図的に局所的なキャパシタンスとインダクタンスを設けることでLC共振器をつくり出している。この場合共振周波数は回路の局所的な共振パラメータにより求まり、線路長には依らない。
        % 共振器のパラメータ計算には手計算に依る解析的な方法と電磁界シミュレーションに依る2つの方法を用いた。電磁界シミュレーションにはAWR社のマイクロウェーブオフィス(ver14.03)を使用した。
    % \subsection{電磁界シミュレーション}
    %     共振器の1次結合は電磁界シミュレータ―を用いて算出した。シミュレーションした構造は以下の2つである。
    %     \begin{figure}[H]
    %         \subfigure{
    %         \includegraphics[width=0.5\columnwidth]{resonator_design_for_mwo.pdf}
    %         }
    %         \subfigure{
    %         \includegraphics[width=0.5\columnwidth]{coupled_resonator_mwo.pdf}
    %         }
    %         \caption{電磁界シミュレーションに使用した構造}
    %     \end{figure}
    %     また、電磁界シミュレーションでは電流分布も表示することが可能である。
    %     図中右側の構造は左の構造を左右対象に配置したものである。すなわち共振周波数は左右の共振器で同一になっている。この構造を電磁界シミュレーションすると図中左の構造物をBare Resonaotr図中右側の構造物を左からResonator A、Resonator Bと表現する。Resonator AとResonator BはBare Resonatorを中点としてそれぞれ結合強度の1/2で反発することが予想される。このシミュレーションによってもっと待った共振周波数の差は共振器間の1次結合gに対応する。
        \begin{figure}[H]
            \centering
            \includegraphics[width=12cm]{mwo.pdf}
            \caption{dc-SQUIDによる$\beta$変調}
        \end{figure}
        青、赤が結合モードの共振周波数、緑が単一モードの共振周波数である。結果からわかる通り、単一モードの共振周波数を中心にモードシフトが生じている。この結果から本回路デザインでは約534MHzで1次結合していることがわかる。
        電磁界シミュレーションによって算出された共振周波数を以下に示す。
        \begin{table}[H]
            \caption{共振周波数}
            \centering
            \begin{tabular}{@{}cc@{}}
            \toprule
                           & Frequency [GHz] \\ 
                           \hline \hline
            Resonator A    & 6.6700          \\
            Resonaotor B   & 7.7382          \\
            Bare Resonator & 7.1624          \\ \bottomrule
            \end{tabular}
            \end{table}
        次に共振器のキャパシタンス、インダクタンス及び共振器を流れる電流の値を推定する方法を解説する。
        既に解説したように、電磁界シミュレーションでは諸所の物理量の計算ができる。ここでは、アドミッタンスYを考える。
        共振周波数$\omega_0=1/\sqrt{LC}$を持った直列回路に於ける入力インピーダンス$Z_{in}$は
        \begin{equation}
            Z_{\text {in }}=R+j\left(\omega L-\frac{1}{\omega C}\right)
        \end{equation}
        アドミタンスは抵抗の逆数であるためその虚部は
        \begin{subequations}
            {\jot=10pt
            \begin{eqnarray}
                Im Y_{in} &=& \frac{1}{\omega L + \frac{1}{\omega C}}\\
                & = & \frac{1}{\omega L \biggl(1+\frac{1}{\omega^2 L C}\biggr)}\\
                & = & \frac{1}{\omega L \biggl(1+\frac{\omega_0^2}{\omega^2}\biggr)}\\
            \end{eqnarray}
            }
        \end{subequations}
        ここで$\omega<<\omega_0$という条件を加えるとアドミッタンスYの虚部は
        \begin{equation}
            ImY = \omega C
        \end{equation}
        となり、キャパシタンスCを比例定数とした直線が得られる。よってこれを線形フィットすることで回路のキャパシタンスを推定することができる。回路のキャパシタンスが推定できれば、共振周波数を用いてインダクタンス、ZPFの電流$I_{ZPF}$まで推定することができる。また、1次結合の値から共振器間の相互インダクタンスも推定が可能である。
        結果をまとめたものを以下に示す。
        \begin{table}[H]
            \caption{結合共振器のパラメータ}
            \centering
            \begin{tabular}{@{}cc@{}}
            \toprule
            Parameter & Value    \\ 
            \hline \hline
            Current        & 5.740 nA \\
            Mutual Inductance       & 0.107 nH \\ 
            coupling constant         & 534 MHz  \\\bottomrule
            \end{tabular}
        \end{table}
        電磁界シミュレーターを用いてもう一つ物理量を推定する。補足ではミアンダインダクタンスを解析的な手法で求める手順を記載した。ここでは、電磁界シミュレーションを用いてミアンダインダクタンスを推定する。先ほどはミアンダインダクタンスを含んだ単一共振器の電磁界シミュレーションを行った。次に計算するのはミアンダインダクタンスを取り除いた単一共振器である。
        \begin{figure}[H]
            \centering
            \includegraphics[width=12cm]{bareresonator_zoom.pdf}
            \caption{ミアンダインダクタンスを含まない単一共振器}
            \label{ミアンダなし}
        \end{figure}
        図\ref*{単一共振モード}\ref*{ミアンダなし}はミアンダの可否以外は全く同じ構造である、よってその差分がミアンダ構造に依る者であると考えればある程度の推定ができる。
        \begin{figure}[H]
            \centering
            \includegraphics[width=12cm]{meanderinductance.pdf}
            \caption{電磁界シミュレーションによる導出}
        \end{figure}
        補足に記したミアンダの本数Nに対して共振周波数の変化をプロットしたものが以下である。
        \begin{figure}[H]
            \centering
            \includegraphics[width=12cm]{mwo_meander.pdf}
            \caption{dc-SQUIDによる$\beta$変調}
        \end{figure}
        先ほどの低周波シュミレーションの方法と同様の手法で共振回路のインダクタンスを計算する。次にミアンダなしのインダクタンスとの差分をとることで、ミアンダの本数に対するミアンダインダクタンスの値を計算することができる。その結果を以下に示す。
        \begin{figure}[H]
            \centering
            \includegraphics[width=12cm]{simulatorVSanalytical.pdf}
            \caption{Inductance VS Meandering Number}
        \end{figure}
        % これによりシミュレータと解析的な計算によるミアンダインダクタンスには大きな差異はなかったことがわかる。本サンプルの設計にはミアンダの回数を38本に設定した。これは解析解とシミュレータに依る解が最も小さな本数を選んだ。また、インダクタンスが大きすぎると共振器間の距離にも依存するが、測定環境である4GHz~8GHz帯を逸脱してしまうため、外部磁場に依る遷移周波数の変化も考慮するとN=38本程度が妥当であるという帰結である。以下にN=38に置ける共振器埜パラメータを示す。

        % さて、作成したサンプルであるが、メインループとdc-SQUIDの面積比を100倍にすることで独立に操作できるようにしたと述べたが、実際にはどのような応答が得られるを示す。オンチップバイアスを固定した状態でグローバルバイアスを適当な値だけ駆動するとdc-SQUIDの応答をみることができる。
        ここまでが共振器の電磁界シミュレーション及び電磁界シミュレーターを用いたパラメータ推定の方法である。最後にそれぞれのサンプルについて電磁界シミュレーションによって得られた設計値を表にしてまとめる。次節ではそれぞれのサンプルについて詳細な説明を行う。
\section{測定サンプル}
    初期、中期、後期サンプルについて詳細な説明を行う。それぞれのサンプルの特徴的な点は既に第1節で記載した。初期サンプルについては有意な結果がえられたのがLEAlのみであったのでここではLEAlのみ解説を行う。
    \subsection{LEAl}
    \begin{figure}[H]
        \centering
        \includegraphics[width=14cm]{design1.pdf}
        \caption{LEAl回路デザイン}
    \end{figure}
    パラメーター――――――――――――――――――――――――――――――\\
    ーーーーーーーーーーーーーーーーーーーーーー\\
    ーーーーーーーーーーーーーーーーーーーーー\\
    ーーーーーーーーーーーーーーーーーーーーーー\\
    ーーーーーーーーーーーーーーーーーーーーー\\
    \subsection{LEJ}
    \begin{figure}[H]
        \centering
        \includegraphics[width=14cm]{design2.pdf}
        \caption{LEJ回路デザイン}
    \end{figure}
    パラメーター――――――――――――――――――――――――――――――\\
    ーーーーーーーーーーーーーーーーーーーーーー\\
    ーーーーーーーーーーーーーーーーーーーーー\\
    ーーーーーーーーーーーーーーーーーーーーーー\\
    ーーーーーーーーーーーーーーーーーーーーー\\
    \subsection{LEM}
    \begin{figure}[H]
        \centering
        \includegraphics[width=14cm]{design3.pdf}
        \caption{LEM回路デザイン}
        \label{LEM}
    \end{figure}
        \begin{figure}[H]
            \centering
            \includegraphics[width=12cm]{compound1.pdf}
            \caption{dc-SQUIDによる$\beta$変調}
        \end{figure}
        図中黒線がスクリーニングである。スクリーニングパラメータがゼロとなるとき、結合強度中
        \begin{equation}
            g(\Phi) = g0\frac{\beta cos(\Phi)}{1+\beta cos(\Phi)} + g1
        \end{equation}
        磁束依存項がゼロとなり、結合強度は共振器間の1次結合のみとなる。この時2つのモードは電磁界シミュレーションで計算した周波数帯に対応するシグナルとなる。測定ではこのシミュレーションと同じようにグローバルバイアスのみを駆動しスクリーニングパラメータの区間周期を見積もる。
        次にある点でグローバルバイアスを固定しオンチップバイアスのみを駆動する。これによスクリーニングパラメータを固定下状態で結合素子を操作することが可能となる。


\chapter{測定}
    \begin{abstract}
        測定環境の説明。希釈冷凍機、使用した実験機器など
    \end{abstract}
    \section{測定環境}
    \begin{figure}[H]
        \centering
        \includegraphics[width=11.5cm,angle=-90]{fridgesetup2.pdf}
        \caption{希釈冷凍機のセットアップs}
    \end{figure}


\chapter{結果}
    \begin{abstract}
        測定結果など
    \end{abstract}
    \section{周波数領域測定}
    \subsection{セットアップ}
    \subsection{OnetoneSpectroscopy}
        まずはdc-SQUIDの応答を確かめるために以下のセットアップにてonetonespectroscopyを行った。
        \begin{figure}[H]
            \centering
            \includegraphics[width=14cm]{measurement_06545.pdf}
            \caption{測定結果1}
        \end{figure}

        \begin{figure}[H]
            \centering
            \includegraphics[width=14cm]{measurement_0654234.pdf}
            \caption{測定結果2}
        \end{figure}

\section{時間領域測定}
    \subsection{セットアップ}


\chapter{考察}
    \begin{abstract}
        解析手法の説明と測定結果からいえる結合性能について言及
    \end{abstract}
    \section{結合性能}

\section{結合振動子の物理}



\chapter{結論}
    \begin{abstract}
        結合素子として使えるのかどうか総論
    \end{abstract}
    \section{解析を終えて}


\chapter{展望}
    \begin{abstract}
        今後改善可能性のある部分について言及
    \end{abstract}
    \section{今後の展望}


\chapter{謝辞}
    \begin{abstract}
        謝辞
    \end{abstract}
    \section*{謝辞}
本研究を進めるにあたってお世話になった方々にこの場を借りて御礼申し上げます。
指導教員の蔡教授には日々の研究指導並びに充実した研究環境を与えてくださったこと厚く感謝しております。誠にありがとうございました。
共同研究者の向井寛人氏、朝永顕成氏、伊藤輝氏には日々の充実した議論をしていただき研究を推し進めることができました。ありがとうございました。
特に、朝永顕成氏には研究室配属から卒業までサンプル作成及び研究活動において大変サポートしていただきました。ありがとうございました。

最後に日々活発な議論をしてくださった蔡研究室の皆様、研究活動をサポートしてくださった皆様に感謝申し上げます。ありがとうございました。


This paper is based on results obtained from a project commissioned by the New Energy and
Industrial Technology Development Organization (NEDO). Supports from CREST, JST (Grant
No. JPMJCR1676), and ImPACT Program of Council for Science, Technology and Innovation
(Cabinet Office, Government of Japan) are also appreciated.

\chapter{補足}
    \begin{abstract}
        本文に直接記載すると煩雑になりがちだが重要な計算をここに記す。
    \end{abstract}
    \section{ハミルトニアンの基底変換}
ここでは、Hamiltonianの基底変換を行う。
\begin{eqnarray}
    \hat{\mathcal{H}}&=&\hbar\biggl(\hat{a}^{\dagger }\ \hat{b}^{\dagger }\biggr)\biggl(\begin{array}{cc}
    \tilde{\omega}_{a} & g_{eff}(\Phi ) \\
    g_{eff}(\Phi ) & \tilde{\omega}_{b}
    \end{array}\biggr)\biggl(\begin{array}{l}
    \hat{a} \\
    \hat{b}
    \end{array}\biggr)\\ \\
    &=& \hbar \tilde{\omega}_{a} \hat{a}^{\dagger} \hat{a}+\hbar \tilde{\omega}_{b} \hat{b}^{\dagger} \hat{b}+\hbar g_{eff}(\Phi)\biggl(\hat{a}^{\dagger}\hat{b}+ \hat{b}^{\dagger}\hat{a}\biggr)
\end{eqnarray}
ここで共振器A,Bのそれぞれの生成消滅演算子$\hat{a}{\dagger},\hat{a},\hat{b}{\dagger},\hat{b},$を結合モードの生成消滅演算子$\hat{c}_{+}^{\dagger},\hat{c}_{+},\hat{c}_{-}^{\dagger},\hat{c}_{-}$で変換する。
%演算子
\begin{equation}
    \hat{c}_{\pm}=\frac{\hat{a} \pm \hat{b}}{\sqrt{2}} ,\quad \hat{c}_{\pm}^{\dagger}=\frac{\hat{a}^{\dagger} \pm \hat{b}^{\dagger}}{\sqrt{2}}
\end{equation}
結合モードの生成消滅で元の演算子を書き表すと
\begin{equation}
    \hat{a}=\frac{\hat{c}_{+}+\hat{c}_{-}}{\sqrt{2}} ,\quad \hat{b}=\frac{\hat{c}_{+}-\hat{c}_{-}}{\sqrt{2}}
\end{equation}
\begin{equation}
    \hat{a}^{\dagger}=\frac{\hat{c}_{+}^{\dagger}+\hat{c}_{-}^{\dagger}}{\sqrt{2}} ,\quad \hat{b}^{\dagger}=\frac{\hat{c}_{+}^{\dagger}-\hat{c}_{-}^{\dagger}}{\sqrt{2}}
\end{equation}
となる。よってその演算子の積は
%積
\begin{equation}
    \hat{a}^{\dagger} \hat{a}=\frac{1}{2}\biggl(\hat{c}_{+}^{\dagger}+\hat{c}_{-}^{\dagger}\biggr) \biggl(\hat{c}_{+}+\hat{c}_{-}\biggr) ,\quad \hat{a}^{\dagger} \hat{b}=\frac{1}{2}\biggl(\hat{c}_{+}^{\dagger}+\hat{c}_{-}^{\dagger}\biggr)\biggl(\hat{c}_{+}-\hat{c}_{-}\biggr)
\end{equation}
\begin{equation}
    \hat{b}^{\dagger} \hat{b}=\frac{1}{2}\biggl(\hat{c}_{+}^{\dagger}-\hat{c}_{-}^{\dagger}\biggr)\biggl(\hat{c}_{+}-\hat{c}_{-}\biggr) ,\quad \hat{b}^{\dagger}\hat{a} = \frac{1}{2}\biggl(\hat{c}_{+}^{\dagger}-\hat{c}_{-}^{\dagger}\biggr)\biggl(\hat{c}_{+}+\hat{c}_{-}\biggr)
\end{equation}
と表現できる。それぞれ計算を進めると

\begin{equation}
    \hat{a}^{\dagger} \hat{a}=\frac{1}{2}\biggl(\hat{c}_{+}^{\dagger} \hat{c}_{+}+\hat{c}_{+}^{\dagger} \hat{c}_{-}+\hat{c}_{-}^{\dagger} \hat{c}_{+}+\hat{c}_{-}^{\dagger} \hat{c}_{-}\biggr)
\end{equation}
\begin{equation}
    \hat{b}^{\dagger} \hat{b}=\frac{1}{2}\biggl(\hat{c}_{+}^{\dagger} \hat{c}_{+}-\hat{c}_{+}^{\dagger} \hat{c}_{-}-\hat{c}_{-}^{\dagger} \hat{c}_{+}+\hat{c}_{-}^{\dagger} \hat{c}_{-}\biggr)
\end{equation}

\begin{equation}
    \hat{a}^{\dagger} \hat{b}=\frac{1}{2}\biggl(\hat{c}_{+}^{\dagger}\hat{c}_{+}-\hat{c}_{+}^{\dagger} \hat{c}_{-}+\hat{c}_{-}^{\dagger} \hat{c}_{+}-\hat{c}_{-}^{\dagger} \hat{c}_{-}\biggr)
\end{equation}
\begin{equation}
    \hat{b}^{\dagger}\hat{a} =\frac{1}{2}\biggl(\hat{c}_{+}^{\dagger}\hat{c}_{+} +\hat{c}_{+}^{\dagger}\hat{c}_{-} - \hat{c}_{-}^{\dagger}\hat{c}_{+}-\hat{c}_{-}^{\dagger}\hat{c}_{-}\biggr)
\end{equation}

これを元のハミルトニアンに代入すれば

\begin{equation}
    \hat{H}=\frac{\hbar}{2} \hat{\omega}_{a}\biggl(\hat{c}_{+}^{\dagger} \hat{c}_{+}+\hat{c}_{+}^{\dagger} \hat{c}_{-}+\hat{c}_{-}^{\dagger} \hat{c}_{+}+\hat{c}_{-}^{\dagger} \hat{c}_{-}\biggr)+\frac{\hbar}{2} \hat{\omega}_{b}\biggl(\hat{c}_{+}^{\dagger} \hat{c}_{+}-\hat{c}_{+}^{\dagger} \hat{c}_{-}-\hat{c}_{-}^{\dagger} \hat{c}_{+}+\hat{c}_{-}^{\dagger} \hat{c}_{-}\biggr)
\end{equation}
\begin{equation}
    +\frac{\hbar}{2} g_{eff}\biggl(\hat{c}_{+}^{\dagger}\hat{c}_{+}-\cancel{\hat{c}_{+}^{\dagger} \hat{c}_{-}}+\bcancel{\hat{c}_{-}^{\dagger} \hat{c}_{+}}-\hat{c}_{-}^{\dagger} \hat{c}_{-}+\hat{c}_{+}^{\dagger}\hat{c}_{+} +\cancel{\hat{c}_{+}^{\dagger}\hat{c}_{-}} - \bcancel{\hat{c}_{-}^{\dagger}\hat{c}_{+}}-\hat{c}_{-}^{\dagger}\hat{c}_{-}\biggr)
\end{equation}

式を整理すると

\begin{equation}
    \hat{H}=\frac{\hbar}{2}\biggl(\hat{\omega}_{a}+\hat{\omega}_{b}+2 g(\Phi)\biggr) \hat{c}_{+}^{\dagger} \hat{c}_{+}+\frac{\hbar}{2}\biggl(\hat{\omega} a+\hat{\omega}_{b}-2 g(\Phi)\biggr) \hat{c}_{-}^{\dagger} \hat{c}_{-}
\end{equation}
\begin{equation}
    +\frac{\hbar}{2}\biggl(\hat{\omega}_{a}-\hat{\omega}_{b}\biggr)\biggl(\hat{c}_{+}^{\dagger}\hat{c}_{-}-\hat{c}_{-}^{\dagger} \hat{c}_{+}\biggr)
\end{equation}
ここで第1項と第2項を
\begin{equation}
    \Omega_{+}=\frac{\tilde{\omega}_{a}+\tilde{\omega}_{b}}{2} + g_{eff}(\Phi)
\end{equation}
\begin{equation}
    \Omega_{-}=\frac{\tilde{\omega}_{a}+\tilde{\omega}_{b}}{2} - g_{eff}(\Phi)
\end{equation}
第3項を
\begin{equation}
    \Delta = \frac{\tilde{\omega}_{a}-\tilde{\omega}_{b}}{2}
\end{equation}
とすれば元のハミルトニアンは
\begin{eqnarray}
    \hat{H}&=&\hbar\Omega_{+} \hat{c}_{+}^{\dagger} \hat{c}_{+}+\hbar\Omega_{-} \hat{c}_{-}^{\dagger} \hat{c}_{-} +\hbar \Delta \biggl(\hat{c}_{+}^{\dagger}\hat{c}_{-}-\hat{c}_{-}^{\dagger} \hat{c}_{+}\biggr)\\ \\ 
    &=& \hbar\biggl(\begin{array}{cc}
        \hat{c}_{+}^{\dagger} & \hat{c}_{-}^{\dagger}
        \end{array}\biggr)\biggl(\begin{array}{cc}
        \Omega_{+} & \Delta \\
        \Delta & \Omega_{-}
        \end{array}\biggr)\biggl(\begin{array}{l}
        \hat{c}_{+} \\
        \hat{c}_{-}
        \end{array}\biggr)
\end{eqnarray}

のように結合モードの基底で書き表すことができる。

\section{rf-SQUIDの相互インダクタンス}
dc-SQUIDのインダクタンスは
\begin{equation}
    L_{s}(\Phi)=\frac{\Phi_0}{4\pi I_{c}|{\cos({\phi_{-}^{min}(\Phi_{ext})}})|}
\end{equation}
と記述することができる。
\begin{equation}
    \Phi=\Phi_{ext}+L_{loop}
\end{equation}
\begin{equation}
    \beta_{dc}=\frac{2\pi L_{loop} I_{c}}{\Phi_{0}}
\end{equation}
とすると。
\section{インターデジタルキャパシタンス}
インターデジタルキャパシタンスとは図中の共振器の櫛状になっている部分の構造である。
\begin{figure}[H]
    \label{le}
    \centering
    \includegraphics[width=16cm]{lumpedelement.pdf}
    \caption{準集中定数型共振器(再掲)}
\end{figure}
この構造により、電極同士の表面積を上げることでキャパシタンスを増幅することができる。ここではインターデジタルキャパシタンスの計算方法について解説を行う。また、ここで解説するインターデジタルキャパシタンスの計算方法は櫛の数が2本以上のケースである。\cite*{Gevorgian1996,Dib2005,Dib2001ComprehensiveSO}
\begin{figure}[H]
    \centering
    \includegraphics[width=14cm]{IDC2.pdf}
    \caption{インターデジタルキャパシタンス}
\end{figure}
式中の文字は上記の図中のパラメータに対応している。
まず線路の幅wは導体の厚さを含めた実行幅$w_eff$へと変換を行う。\cite*{Gevorgian1996}
\begin{equation}
    W_{e f f}=W+\frac{t}{\pi}\biggl(1+\ln \biggl(\frac{4 \pi W}{t}\biggr)\biggr)
\end{equation}
ここで相対する導体間のキャパシタンスをCsとする。
キャパシタンスCsを構成するのは櫛が3本であることを想定したキャパシタンスC3、櫛の本数Nに相当するキャパシタンス開放状態になっている最両端の櫛のキャパシタンスCendの3つである。
\begin{equation}
    C_{s}=C_{3}+C_{N}+C_{\text { end }}
\end{equation}
それぞれのキャパシタンスはコンフォーマルマッピングを用いて解析的に求めることが可能である。それぞれの値を求めるにはまずC3について
\begin{equation}
    C_{3}=4 \varepsilon_{0} \varepsilon_{e} \frac{K\biggl(k_{1}\biggr)}{K\biggl(k_{1}^{\prime}\biggr)} L
\end{equation}\begin{equation}
    \varepsilon_{e}=1+q_{1} \frac{\varepsilon_{r}-1}{2}
\end{equation}
\begin{equation}
        q_{1}=\frac{K\biggl(k_{1}^{\prime}\biggr)}{K\biggl(k_{1}\biggr)} \frac{K\biggl(k_{2}\biggr)}{K\biggl(k_{2}^{\prime}\biggr)}
\end{equation}
\begin{equation}
    k_{1}=\frac{W}{W+2 S} \sqrt{\frac{1-\biggl(\frac{W+2 S}{3 W+2 S}\biggr)^{2}}{1-\biggl(\frac{W}{3 W+2 S}\biggr)^{2}}}
\end{equation}
\begin{equation}
    \begin{aligned}
    k_{2}=& \frac{\sinh \biggl(\frac{\pi W}{4 h}\biggr)}{\sinh \biggl(\frac{\pi(W+2 S)}{4 h}\biggr)} 
    & \sqrt{\frac{\sinh ^{2}\biggl(\frac{\pi(3 W+2 S)}{4 h}\biggr)-\sinh ^{2}\biggl(\frac{\pi(W+2 S)}{4 h}\biggr)}{\sinh ^{2}\biggl(\frac{\pi(3 W+2 S)}{4 h}\biggr)-\sinh ^{2}\biggl(\frac{\pi W}{4 h}\biggr)}}
    \end{aligned}
\end{equation}
同様にしてCNについて求めると
\begin{equation}
    C_{N}=(N-3) \varepsilon_{0} \varepsilon_{N} \frac{K\biggl(k_{3}\biggr)}{K\biggl(k_{3}^{\prime}\biggr)} L
\end{equation}
\begin{equation}
    \varepsilon_{N}=1+q_{N} \frac{\varepsilon_{r}-1}{2}
\end{equation}
\begin{equation}
    q_{N}=\frac{K\biggl(k_{3}^{\prime}\biggr)}{K\biggl(k_{3}\biggr)} \frac{K\biggl(k_{4}\biggr)}{K\biggl(k_{4}^{\prime}\biggr)},
\end{equation}
\begin{equation}
    k_{3}=\frac{W}{W+S},
\end{equation}
\begin{equation}
    \begin{aligned}
    k_{4}=& \frac{\sinh \biggl(\frac{\pi W}{4 h}\biggr)}{\sinh \biggl(\frac{\pi(W+ S)}{4 h}\biggr)} 
    & \sqrt{\frac{\sinh ^{2}\biggl(\frac{\pi(W+S)}{4 h}\biggr)+\sinh ^{2}\biggl(\frac{\pi(W+ S)}{4 h}\biggr)}{\cosh ^{2}\biggl(\frac{\pi(W)}{4 h}\biggr)-\sinh ^{2}\biggl(\frac{\pi (W+S)}{4 h}biggrt)}}
    \end{aligned}
\end{equation}
として求まる。最後にCendについて、この計算を行う上で主に参考にしている論文\cite*{Dib2005}ではCendの計算について\cite*{Dib2001ComprehensiveSO}中の式
\begin{equation}
    C_{o c}=c_{e f f} C_{o e}\biggl(\epsilon_{r}=1\biggr)
\end{equation}
\begin{equation}
    \label{Cend}
    C_{\alpha c}\biggl(c_{r}=1\biggr)=\frac{c_{0}}{\pi}\biggl(\frac{1}{g^{2}} f_{s}(g, W+S, 0)+\frac{1}{W^{2}} f_{0}(W, S, 0)\biggr)
\end{equation}
\begin{equation}
    \label{fs}
    \begin{aligned}
    f_{s}(a, b, c) &=\frac{4}{3} c^{3}+f(a, c)+f(b, c)-4 a b c \tan ^{-1}\biggl(\frac{a b}{c \tau}\biggr)-\frac{2}{3}\biggl(b^{2}-2 c^{2}+a^{2}\biggr) \tau \\
    &+\biggl(a^{2}-c^{2}\biggr) b \ln \biggl(\frac{\tau+b}{\tau-b}\biggr)+\biggl(b^{2}-c^{2}\biggr) a \ln \biggl(\frac{\tau+a}{\tau-a}\biggr)
    \end{aligned}
\end{equation}
\begin{equation}
    \label{f0}
    f_{0}(a, b, c) =\frac{4}{3} c^{3}+f(a, c)+f(a+b, c)-\frac{1}{2} f(b+2 a, c)-\frac{1}{2} f(b, c)
\end{equation}
\begin{equation}
    \label{fx}
    f(x, y) =\frac{2}{3}\biggl(x^{2}-2 y^{2}\biggr) \sqrt{x^{2}+y^{2}}+y^{2} x \ln \biggl(\frac{\sqrt{x^{2}+y^{2}}+x}{\sqrt{x^{2}+y^{2}}-x}\biggr)
\end{equation}
を使用したと記述されているが式\ref*{fx}の第2項目は不適切であるため、本稿では含めずに計算した。というのも式\ref*{fx}が使用されている式\ref{Cend,fs,f0}に注目すると式\label{fx}中の$y$は常にゼロであり、第2項目は2乗でゼロに収束する。よって2項目は計算に含まれないことになる。よって計算には以下
\begin{equation}
    \label{fx_re}
    f(x, y)_{revised} =\frac{2}{3}\biggl(x^{2}-2 y^{2}\biggr) \sqrt{x^{2}+y^{2}}
\end{equation}
をもちいた。
以上の計算式を用いてインターデジタルキャパシタンスの値を見積もることができる。
ここではさらに本稿で使用したパラメータを用いて櫛の本数Nに対してキャパシタンスの値がどれだけ変化するのか、また、実験結果との比較を行う。
\begin{figure}[H]
    \centering
    \includegraphics[width=10cm]{IDC_cal.pdf}
    \caption{キャパシタンス vs 櫛の本数}
\end{figure}
\section{ミアンダインダクタンス}
ミアンダインダクタンスとは図\ref{le}中蛇行した構造を持ったインダクタのことを指している。蛇行させることによりサイズに対して線路長が長くなり、また、相対する相互インダクタンスによりインダクタンスの総和が増加する。ここでは本稿で用いたミアンダインダクタンスの解析的計算方法について解説する。本文中ではここで行った解析的計算の結果と電磁界シミュレーションによるミアンダインダクタンス部分の計算を比較していので参照されたい。
\begin{figure}[H]
    \label{ミアンダ}
    \centering
    \includegraphics[width=14cm]{ミアンダ.pdf}
    \caption{ミアンダインダクタンス}
\end{figure}
以下数式は上図のパラメータに対応する。この計算に主に参考にした論文は\cite*{Stojanovic2004}である。
文献\cite*{Grover}より長方形線路の自己インダクタンスは
\begin{equation}
    L=0.002 l\{\ln [2 l /(w+t)]+0.50049+[(w+t) / 3 l]\}
\end{equation}
である。ここでwは線路幅、tは厚さ、lは線路長である。ミアンダインダクタンスの大部分を占めているのは各セグメントに於ける自己インダクタンスの総和である。すなわち、図中のパラメータによって各セグメントをラベル付けするとミアンダインダクタンス中セルフインダクタンスの寄与は
\begin{equation}
    L_{\text {selftot }}=2 \cdot L_{a}+2 \cdot L_{b}+N \cdot L_{h}+(N+1) \cdot L_{d}
\end{equation}
である。ここでNはセグメントhの本数に対応する。図\ref*{ミアンダ}ではN=10である。参考文献ではいくつかタイプミスが見受けられたため、少々煩雑ではあるが式の全文を記すこととする。
既に記した条件を前提とした上でNがの偶奇で計算は異なる。
\begin{figure}[H]
    \label{偶奇}
    \centering
    \includegraphics[width=14cm]{偶奇.pdf}
    \caption{ミアンダインダクタンス}
\end{figure}
文献\cite*{Grover}に依れば相対する線路に於ける相互インダクタンスは
\begin{equation}
    M=0.002 l\biggl(\log _{e}\biggl(\frac{l}{d}+\sqrt{1+\frac{l^{2}}{d^{2}}}\biggr)-\sqrt{1+\frac{d^{2}}{l^{2}}}+\frac{d}{l}\biggr)
\end{equation}
\begin{figure}[H]
    \label{相互}
    \centering
    \includegraphics[width=8cm]{相互.pdf}
    \caption{2線路間に於ける相互インダクタンス}
\end{figure}
線路間の相互インダクタンスは配置によって以下の4通りのケースが存在する。
\begin{figure}[H]
    \centering
    \includegraphics[width=14cm]{mutual.pdf}
    \caption{2線路間に於ける相互インダクタンス}
\end{figure}
よってそれぞれについて相互インダクタンスの関数を以下のように定義する。
\begin{equation}
    M_{a 1}\biggl(l_{1}, l_{2}, r, s\biggr)=0.5 \cdot\biggl(M_{c}\biggl(l_{1}+l_{2}+s, r\biggr)+M_{c}(s, r)-M_{c}\biggl(l_{1}+s, r\biggr)-M_{c}\biggl(l_{2}+s, r\biggr)\biggr)
\end{equation}
\begin{equation}
    M_{a 2}\biggl(l_{1}, l_{2}, r\biggr)=0.5 \cdot\biggl(M_{c}\biggl(l_{1}, r\biggr)+M_{c}\biggl(l_{2}, r\biggr)-M_{c}\biggl(l_{1}-l_{2}, r\biggr)\biggr)
\end{equation}
\begin{equation}
    M_{a 3}\biggl(l_{1}, l_{2}, r\biggr)=0.5\biggl(M_{c}\biggl(l_{1}+l_{2}, r\biggr)-M_{c}\biggl(l_{1}, r\biggr)-M_{c}\biggl(l_{2}, r\biggr)\biggr)
\end{equation}
\begin{equation}
    M_{b}\biggl(l_{1}, l_{2}, s\biggr)=\frac{\infty_{0}}{4 \pi}\biggl(\biggl(l_{1}+l_{2}+s\biggr) \ln \biggl(l_{1}+l_{2}+s\biggr)-\biggl(l_{1}+s\biggr) \ln \biggl(l_{1}+s\biggr)-\biggl(l_{2}+s\biggr) \ln \biggl(l_{2}+s\biggr)+s \ln (s)\biggr)
\end{equation}

上記の4通りの方法を各セグメントに適用することでミアンダインダクタンスに存在する相互インダクタンスをすべて勘定することができる。
偶奇それぞれについて計算する相互インダクタンスは以下の7ケースである。
\begin{figure}[H]
    \centering
    \includegraphics[width=14cm]{m2.pdf}
    \caption{セグメント別相互インダクタンス}
\end{figure}
上記のセグメント別の相互インダクタンスについて偶奇について違いあるものは
\begin{equation}
    \begin{array}{l}
    M_{1}=\sum_{i=1}^{N / 2}(2 N+4-4 i) \cdot M_{\mathrm{ul}}(d, d, h,(2 i-2) d), \text { for } N \text { even } \\
    \qquad M_{1}=\sum_{i=1}^{d}(2 N+4-4 i) \cdot M_{a 1}(d, d, h,(2 i-2) d), \text { for } N \text { odd }
    \end{array}
\end{equation}
\begin{equation}
    \begin{aligned}
    M_{2} &=\sum_{i=1}^{N / 2}(2 N+2-4 i) \cdot M_{b}(d, d,(2 i-1) d), \text { for } N \text { even } \\
    M_{2}=&\sum_{(N-1) / 2}^{\infty}(2 N+2-4 i) \cdot M_{b}(d, d,(2 i-1) d), \text { for } N \text { odd }
    \end{aligned}
\end{equation}
\begin{equation}
    \begin{aligned}
    M_{6}&=-2 \cdot M_{c}(b,(N+1) d), \text { for } N \text { even }\\
    M_{6}&=+2 \cdot M_{a 3}(b, b,(N+1) d), \text { for } N \text { odd }
    \end{aligned}
\end{equation}
また、偶奇の違いがないものについて
\begin{equation}
    M_{3}=2 \cdot M_{b}(a, a,(N+1) d)
\end{equation}
\begin{equation}
    M_{4}=\sum_{i=0}^{N} 4 \cdot M_{a 1}(a, d, b, i d)
\end{equation}
\begin{equation}
    M_{5}=\sum_{i=0}^{N-1}(-1)^{i} \cdot 2 \cdot(N-1) \cdot M_{c}(h, t d)
\end{equation}
\begin{equation}
    M_{7}=\sum_{i=0}^{N}(-1)^{i} \cdot 4 \cdot M_{a 2}\biggl(b_{2} h, i d\biggr)
\end{equation}
となる。それぞれについて各セグメントの総和
\begin{equation}
    M_{\text {tot }}=M_{1}+M_{2}+M_{3}+M_{4}+M_{5}+M_{6}+M_{7}
\end{equation}
によってミアンダインダクタンスに於ける相互インダクタンスを見積もることができる。これについて自己インダクタンスを足し合わせたもの
\begin{equation}
    L_{\text {tot }}=L_{\text {selftot }}+M_{\text {tot }}
\end{equation}
がミアンダインダクタンスとなる。ただし、本稿では線路として極低温下にある超伝導微細薄膜を用いているため線路長分のカイネティックインダクタンスを考慮する必要がある。


\section{マスター方程式}
\section{2点相関関数}
\section{等価回路}
本文中で電気回路に於ける基本的な回路の等価変換を行った。ここでは、2端子対回路に於ける等価回路について解説する。\cite*{電気回路,続電気回路の基礎}
電気回路を考える場合入力と出力を考え、上図のように定めると都合がよい。この回路網が満たす条件をここでは
①内部に電源を含まない。ただし、トランジスタの直流電源は差し支えない。
②線形であっても重ね合わせの原理が成立する。
③入力埜1端子から流れ込んだ電流は入力の他端子から流れ出る。出力側も同様である。
2端子対回路は幾通りかの特性表示の方法があり、ここではFパラメータを用いる。Fパラメータを用いる場合の2端子対回路の電流電圧は上図のように表現される。
\begin{figure}[H]
    \centering
    \includegraphics[width=8cm]{2端子.pdf}
    \caption{2端子対マトリックス}
\end{figure}
この場合Fマトリックスは次式で与えられる。
\begin{eqnarray}
    V_1 &= AV_2 + BI_2\\
    I_1 &= CV_2 +DI_2 
\end{eqnarray}
\begin{equation*}
    \begin{pmatrix}V_1\\I_1\end{pmatrix} = \begin{pmatrix}A&B\\C&D\end{pmatrix}\begin{pmatrix}V_2\\I_2\end{pmatrix}
\end{equation*}
\begin{equation*}
    F = \begin{pmatrix}A&B\\C&D\end{pmatrix}
\end{equation*}
この時、行列要素A,B,C,Dについてそれぞれの次元はA(無次元),B($\Omega$),C(S),D(無次元)である。次元Sは伝導度であり、$\Omega$の逆数である。
Fパラメータを用いる理由は複数の2端子回路を組み合わせた時に計算が簡便になるためである。ここで、2端子回路の接続について考える。
\begin{equation*}
    \begin{pmatrix}V_1\\I_1\end{pmatrix} = \begin{pmatrix}A1&B1\\C1&D1\end{pmatrix}\begin{pmatrix}V_2\\I_2\end{pmatrix}
\end{equation*}
\begin{equation*}
    \begin{pmatrix}V_2\\I_2\end{pmatrix} = \begin{pmatrix}A2&B2\\C2&D2\end{pmatrix}\begin{pmatrix}V_3\\I_3\end{pmatrix}
\end{equation*}
したがって
\begin{equation*}
    \begin{pmatrix}V_1\\I_1\end{pmatrix} = \begin{pmatrix}A1&B1\\C1&D1\end{pmatrix}\begin{pmatrix}A2&B2\\C2&D2\end{pmatrix}\begin{pmatrix}V_3\\I_3\end{pmatrix}
\end{equation*}
\begin{equation*}
    \begin{pmatrix}V_1\\I_1\end{pmatrix} = F1 F2 \begin{pmatrix}V_3\\I_3\end{pmatrix}
\end{equation*}
\begin{equation*}
    \begin{pmatrix}V_1\\I_1\end{pmatrix} = F \begin{pmatrix}V_3\\I_3\end{pmatrix}
\end{equation*}
すなわちFパラメータ表現では、各回路の接続は行列の積として与えられる。
この性質を用いてFパラメータを用いて2通りの2端子対回路を示す。
\subsection{T型回路}
図のようにインピーダンス素子を配置したときFパラメータを用いてどのように表現されるかを示す。
\begin{figure}[H]
    \centering
    \includegraphics[width=10cm]{t型.pdf}
    \caption{T型2端子回路}
\end{figure}
Fパラメータの積の性質及び、各行列要素に適切な値を入力することでT型回路のFパラメータは
\begin{eqnarray}
    \begin{pmatrix}A&B\\C&D\end{pmatrix} &= \begin{pmatrix}1&Z1\\0&1\end{pmatrix}\begin{pmatrix}1&0\\\frac{1}{Z2}&0\end{pmatrix}\begin{pmatrix}1&Z3\\0&1\end{pmatrix}\\
    &=\begin{pmatrix}1+\frac{Z1}{Z2}&Z1+Z3+\frac{Z1Z3}{Z2}\\\frac{1}{Z2}&1+\frac{Z3}{Z2}\end{pmatrix}
\end{eqnarray}
\subsection{$\Pi$型回路}
同様にして$\Pi$型回路についてFパラメータを求めると
\begin{figure}[H]
    \centering
    \includegraphics[width=10cm]{pi型.pdf}
    \caption{$\Pi$型2端子回路}
\end{figure}
\begin{eqnarray}
    \begin{pmatrix}A&B\\C&D\end{pmatrix} &= \begin{pmatrix}1&Z1\\0&1\end{pmatrix}\begin{pmatrix}1&0\\\frac{1}{Z2}&0\end{pmatrix}\begin{pmatrix}1&Z3\\0&1\end{pmatrix}\\
    &=\begin{pmatrix}1+\frac{Z1}{Z2}&Z1+Z3+\frac{Z1Z3}{Z2}\\\frac{1}{Z2}&1+\frac{Z3}{Z2}\end{pmatrix}
\end{eqnarray}
\subsection{$T-\Pi$変換($Y-\Delta$変換)}
以上の2つの回路についてそれぞれのFパラメータを比較することで回路の等価変換を行うことができる。便宜状T型のインピーダンスをZ、$\Pi$型のインピーダンスをXとして表すと
\\
$\Pi$型\\
\begin{eqnarray}
    \begin{pmatrix}A&B\\C&D\end{pmatrix} &= \begin{pmatrix}1&X1\\0&1\end{pmatrix}\begin{pmatrix}1&0\\\frac{1}{X2}&0\end{pmatrix}\begin{pmatrix}1&X3\\0&1\end{pmatrix}\\
    &=\begin{pmatrix}1+\frac{X1}{X2}&X1+X3+\frac{X1X3}{X2}\\\frac{1}{X2}&1+\frac{X3}{X2}\end{pmatrix}
\end{eqnarray}
T型\\
\begin{eqnarray}
    \begin{pmatrix}A&B\\C&D\end{pmatrix} &= \begin{pmatrix}1&Z1\\0&1\end{pmatrix}\begin{pmatrix}1&0\\\frac{1}{Z2}&0\end{pmatrix}\begin{pmatrix}1&Z3\\0&1\end{pmatrix}\\
    &=\begin{pmatrix}1+\frac{Z1}{Z2}&Z1+Z3+\frac{Z1Z3}{Z2}\\\frac{1}{Z2}&1+\frac{Z3}{Z2}\end{pmatrix}
\end{eqnarray}

連立方程式を解くとそれぞれ
\begin{eqnarray}
    X1 &= \frac{Z1Z2+Z2Z3+Z3Z1}{Z3}\\
    X2 &= \frac{Z1Z2+Z2Z3+Z3Z1}{Z2}\\
    X3 &= \frac{Z1Z2+Z2Z3+Z3Z1}{Z1}\\
\end{eqnarray}
また、この逆は
\begin{eqnarray}
    Z1 &= \frac{X1X2}{X1+X2+X3}\\
    Z2 &= \frac{X3X1}{X1+X2+X3}\\
    Z3 &= \frac{X2X3}{X1+X2+X3}\\
\end{eqnarray}
となる。
次にインダクタンスにより磁気的に結合された回路をT型回路によって表示する方法を以下に示す。
\subsection{磁気結合回路}
磁気的に結合された回路を以下のように表す。この回路表記は変圧器の回路表現と同等である。
\begin{figure}[H]
    \centering
    \includegraphics[width=12cm]{磁気的結合回路.pdf}
    \caption{磁気的結合回路}
\end{figure}
Zmは通常相互インダクタンスMを用いて
\begin{equation*}
    Z_m = \pm i\omega M
\end{equation*}
として表現できる。
変圧器のFパラメータを求めるためにキルヒホッフの法則より電圧と電流の関係は
\begin{eqnarray}
    V_1 &= Z_p I_1 -Z_m I_2\\
    -V_2 &= -Z_m I_1 + Z_s I_2\\
\end{eqnarray}
上式を整理して
\begin{eqnarray}
    V_1 &= \frac{Z_P}{Z_M}V_2 - \frac{Z_PZ_S-Z_M^2}{Z_M}I2\\
    I_1 &= \frac{1}{Z_M}V_2+\frac{Z_S}{Z_M}I_2
\end{eqnarray}
行列表現に直すことでFパラメータが
\begin{equation*}
    \begin{pmatrix}A&B\\C&D\end{pmatrix} =\begin{pmatrix}1+\frac{Z_P}{Z_M}&\frac{Z_PZ_S-Z_M^2}{Z_M}\\\frac{1}{Z_M}&1+\frac{Z_S}{Z_M}\end{pmatrix}
\end{equation*}
と求まる。T型と比較すると
\begin{eqnarray}
    Z_1 &= Z_P-Z_M\\
    Z_2 &= Z_M\\
    Z_3 &= Z_S-Z_M
\end{eqnarray}
と表現することができる。インダクタンスLを用いてよりあらわに以上の関係を図式すると
\begin{figure}[H]
    \centering
    \includegraphics[width=12cm]{変圧器.pdf}
    \caption{変圧器とT型回路}
\end{figure}



%\bibliographystyle{}
\addcontentsline{toc}{chapter}{参考文献}
\printbibliography[title=参考文献]
\end{document}
