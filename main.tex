\documentclass[]{jsbook}
\usepackage[top=25truemm,bottom=25truemm,left=25truemm,right=25truemm]{geometry}
\usepackage[dvipdfmx]{graphicx}
\usepackage[dvipdfmx]{color}
\usepackage[dvipdfmx,colorlinks,linkcolor=blue,urlcolor=blue,bookmarks=true,bookmarksnumbered=true]{hyperref}
\usepackage{pxjahyper}
\usepackage{bookmark}
\usepackage{url}
\usepackage{amsmath}
\usepackage{amssymb}
\usepackage{amsfonts}
\usepackage{here}
\usepackage{comment}
%\usepackage{physics}
%\usepackage[style=phys,articletitle=true,biblabel=brackets,chaptertitle=false,pageranges=false,backend=biber]{biblatex}
%\addbibresource{bibs.bib}
% \usepackage{amsmath}
% \usepackage{amssymb}
% \usepackage{amsfonts}
% \usepackage{here}
% \usepackage{comment}
% \usepackage{physics}

\begin{document}
\begin{titlepage}
    \begin{center}
        {\Large 2020年度卒業論文}\\
        \vspace{180truept}
        {\Huge 超伝導準集中定数回路を用いた\\
        \vspace{10truept}
        高強度結合素子sの製作}\\ 
        \vspace{70truept}

        {\Large \today}\\

        \vspace{70truept}

        {\Large 東京理科大学 理学研究科物理学専攻 蔡研究室\\
        (学籍番号 1219537)}\\

        \vspace{20truept}

        {\huge 宮永 崇史}\\

        \vspace{160truept}
        {\Large 東京理科大学 理学s研究s科物理学専攻}\\
    \end{center}
\end{titlepage}
%\maketitle
\frontmatter
\addcontentsline{toc}{chapter}{概要}
\section*{序章}
超伝導体を用いた擬似人工原子システムの確立\cite*{nakamura1999coherent}から約23年を経た現在、超伝導量子計算機の開発に注目が集まっている。量子計算機が古典コンピュータと比較して有意な結果を発揮するためには約100万もの擬似人口原子(以下量子ビット)を集積してからだといわれており、回路の集積度は年々増加している。
回路に装填される素子は量子ビットだけではない。量子情報を取得する読み出し素子。量子ビット同士を相互作用させる結合素子。量子情報を伝送させる伝送ライン。これら素子群をパッケージングする回路デザインも量子ビットの集積度に大きく寄与する。当研究室でも万能型量子計算機、専門特化型量子計算機の回路アーキテクチャをそれぞれ提案しており、2次元平面状で実装可能であることが利点である。
\tableofcontents
\mainmatter

\chapter{sssszs序論}
    \begin{abstract}
        2007年にカナs;ダのD-wave社が大規模な量子アニーリングシステムを開発したことにより世間の注目を浴びた量子計算。その計算手法の構築に日本の研究者が関与したということはよく知られるところである。\\
        この章では従来の量子アニーリング手法について簡便に解説したのち、今福さんによる新手法の意義、方法について述べる。\\

    \end{abstract}
    \section{アニーリング量子計算sss}
基底状態探索として近年量子アニーリングという手法がよく使われている。この計算手法は駆動ハミルトニアンをVと定義したときに系のハミルトニアンを
\begin{equation}
    \hat{H}(t) = s(t) \hat{H} +(1-s(t))\hat{V}
\end{equation}
と定義する。初期状態は$t=0$で$s(t)=0$、$t=t_f$で$s(t_f)=1$であることに注意する。
量子力学の断熱定理によれば系の初期状態を初期ハミルトニアンVで用z意すzれば逐次的にハミルトニaアン
を変動させることで$t=t_f$ではハミルトニアン$H_f$において基底\cite*{Baust2015}状態を実現することができるというものである。

上式によればエネルギーgapが小さければ小さいほ\cite*{Wulschner2016}

\end{document}