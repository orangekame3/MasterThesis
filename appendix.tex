\section{ハミルトニアンの基底変換}
ここでは、Hamiltonianのs変換を行う。
\begin{equation}
    \hat{\mathcal{H}}=\hbar\left(\hat{a}^{\dagger }\ \hat{b}^{\dagger }\right)\left(\begin{array}{cc}
    \tilde{\omega}_{a} & g(\Phi ) \\
    g(\Phi ) & \hat{\omega}_{b}
    \end{array}\right)\left(\begin{array}{l}
    \hat{a} \\
    \hat{b}
    \end{array}\right)
\end{equation}
\begin{equation}
    = \hbar \hat{\omega}_{a} \hat{a}^{\dagger} \hat{a}+\hbar \hat{\omega}_{b} \hat{b}^{\dagger} \hat{b}+\hbar g(\Phi)\left(\hat{a}^{\dagger}\hat{b}+\hat{a} \vec{b}^{+}\right)
\end{equation}
\begin{equation}
    \hat{c}_{\pm}=\frac{\hat{a} \pm \hat{b}}{\sqrt{2}} \quad \hat{c}_{+}^{\dagger}=\frac{\hat{a}^{\dagger} \pm \hat{b}^{\dagger}}{\sqrt{2}}
\end{equation}
\begin{equation}
    \hat{a}^{\dagger}=\frac{\hat{c}_{+}^{\dagger}+\hat{c}_{-}^{\dagger}}{\sqrt{2}} \quad \hat{b}=\frac{\hat{c}_{+}^{\dagger}-\hat{c}_{-}^{\dagger}}{\sqrt{2}}
\end{equation}
\begin{equation}
    \hat{a}=\frac{\hat{c}_{+}+\hat{c}}{\sqrt{2}} \quad \hat{b}=\frac{\hat{c}_{+}-\hat{c}_{-}}{\sqrt{2}}
\end{equation}
\begin{equation}
    \hat{a}^{\dagger} \hat{a}=\frac{1}{2}\left(\hat{c}_{+}^{\dagger}+\hat{c}_{-}^{+}\right)\left(\hat{c}_{t}+\hat{c}_{-}\right) \quad \hat{a}^{\dagger} \hat{b}=\frac{1}{2}\left(\hat{c}_{t}^{+}+\hat{c}_{-}^{+}\right)\left(\hat{c}_{+}-\hat{c}_{-}\right)
\end{equation}
\begin{equation}
    \hat{b} \hat{b}=\frac{1}{2}\left(\hat{c}_{t}+\hat{c}_{-}^{+}\right)\left(\hat{c}_{+}-\hat{c}_{-}\right) \quad \hat{a} \hat{b}^{\dagger}-\frac{1}{2}\left(\hat{c}_{t}+\hat{c}_{-}\right)\left(\hat{c}_{t}^{+}-\hat{c}_{-}^{+}\right)
\end{equation}
\begin{equation}
    \hat{a}^{\dagger} \hat{a}=\frac{1}{2}\left[\hat{c}_{t}^{+} \hat{c}_{+}+\hat{c}_{t}^{+} \hat{c}+\hat{c}_{-}^{+} \hat{c}_{t}+\hat{c}_{-}^{+} \hat{c}_{-}\right]
\end{equation}
\begin{equation}
    \hat{b}^{\dagger} \hat{b}=\frac{1}{2}\left[\hat{c}_{t}^{*} \hat{c}_{t}-\hat{c}_{t}^{n} \hat{c}-\hat{c}_{-}^{+} \hat{c}_{t}+\hat{c}_{-}^{+} \hat{c}-\right]
\end{equation}
\begin{equation}
    \hat{a}^{\dagger} \hat{b}=\frac{1}{2}\left[\dot{c}+\hat{c}_{t}-\hat{c}_{t} \hat{c}_{-}+\hat{c}_{-}^{+} \hat{c}_{t}-\hat{c}_{-}^{\prime} \hat{c}_{-}\right]
\end{equation}
\begin{equation}
    \hat{a} \hat{b}^{\dagger}=\frac{1}{2}\left[\hat{c}_{+} \hat{c}_{+}^{4}-\hat{c}_{+} \hat{c} \pm+\hat{c}-\hat{c}_{+}^{2}-\hat{c}-\hat{c}_{-}\right]
\end{equation}
\begin{equation}
    \hat{H}=\frac{\hbar}{2} \hat{w}_{a}\left[\hat{c}_{t}^{+} \hat{c}_{+}+\hat{c}_{t}^{+} \hat{c}+\hat{c}_{-}^{+} \hat{c}_{t}+\hat{c}_{-}^{+} \hat{c}_{-}\right]
\end{equation}
\begin{equation}
    +\frac{\hbar}{2} \hat{\omega}_{b}\left[\hat{c}_{+}^{+} \hat{c}_{+}-\hat{c}_{+}^{+} \hat{c}_{-}-\hat{c}_{-}^{+} \hat{c}_{+}+\hat{c}_{-}^{+} \hat{c}_{-}\right]
\end{equation}
\begin{equation}
    +\frac{\hbar}{2} g\left[2 \hat{c}+\hat{c}_{+}-2 \hat{c}_{-} \hat{c}\right]
\end{equation}
\begin{equation}
    \hat{H}=\frac{\hbar}{2}\left(\hat{\omega}_{a}+\hat{\omega}_{b}+2 g(\Phi)\right) \hat{c}_{t}^{+} \hat{c}_{+}+\frac{\hbar}{2}\left(\hat{w} a+\hat{w}_{b}-2 g(\Phi)\right) \hat{c}_{-}^{+} \hat{c}_{-}
\end{equation}
\begin{equation}
    +\frac{\hbar}{2} A\left(\hat{w}_{a}-\hat{w}_{b}\right)\left(\hat{c}_{t}+\hat{c}_{-}+\hat{c}_{+} \hat{c}_{-}\right)
\end{equation}
\begin{equation}
    \omega_{a}+\bar{c}_{b}+2 g(\Phi)=\Omega+
\end{equation}
\begin{equation}
    \omega_{a}+\bar{c}_{b}-2 g(\Phi)=\Omega-
\end{equation}
\begin{equation}
    \hat{\omega}_{a}-\hat{w}_{b}=\Delta
\end{equation}
\begin{equation}
    \hat{H}=\frac{\hbar}{2} \Omega+\hat{c}+\hat{c}+\frac{\hbar}{2} \Omega-\hat{c} \pm \hat{c}+\frac{\hbar}{2} \Delta\left(\hat{c}+\hat{c}+\hat{c}_{+} \hat{c}_{-}^{+}\right)
\end{equation}
\begin{equation}
    -\frac{\hbar}{2}\left(\begin{array}{cc}
    \hat{c}_{+} & \hat{c} \pm
    \end{array}\right)\left(\begin{array}{cc}
    \Omega_{1} & \Delta \\
    2 & \Omega_{2}
    \end{array}\right)\left(\begin{array}{l}
    \hat{c}_{+} \\
    \hat{c}_{-}
    \end{array}\right)
\end{equation}

\section{rf-SQUIDの相互インダクタンス}
dc-SQUIDのインダクタンスは
\begin{equation}
    L_{s}(\Phi)=\frac{\Phi_0}{4\pi I_{c}|{\cos({\phi_{-}^{min}(\Phi_{ext})}})|}
\end{equation}
と記述することができる。
\begin{equation}
    \Phi=\Phi_{ext}+L_{loop}
\end{equation}
\begin{equation}
    \beta_{dc}=\frac{2\pi L_{loop} I_{c}}{\Phi_{0}}
\end{equation}
とすると。
\section{インターデジタルキャパシタンス}
インターデジタルキャパシタンスとは図中の共振器の櫛状になっている部分の構造である。
\begin{figure}[H]
    \label{le}
    \centering
    \includegraphics[width=16cm]{lumpedelement.pdf}
    \caption{準集中定数型共振器(再掲)}
\end{figure}
この構造により、電極同士の表面積を上げることでキャパシタンスを増幅することができる。ここではインターデジタルキャパシタンスの計算方法について解説を行う。また、ここで解説するインターデジタルキャパシタンスの計算方法は櫛の数が2本以上のケースである。\cite*{Gevorgian1996,Dib2005,Dib2001ComprehensiveSO}
\begin{figure}[H]
    \centering
    \includegraphics[width=14cm]{IDC2.pdf}
    \caption{インターデジタルキャパシタンス}
\end{figure}
式中の文字は上記の図中のパラメータに対応している。
まず線路の幅wは導体の厚さを含めた実行幅$w_eff$へと変換を行う。\cite*{Gevorgian1996}
\begin{equation}
    W_{e f f}=W+\frac{t}{\pi}\left[1+\ln \left(\frac{4 \pi W}{t}\right)\right]
\end{equation}
ここで相対する導体間のキャパシタンスをCsとする。
キャパシタンスCsを構成するのは櫛が3本であることを想定したキャパシタンスC3、櫛の本数Nに相当するキャパシタンス開放状態になっている最両端の櫛のキャパシタンスCendの3つである。
\begin{equation}
    C_{s}=C_{3}+C_{N}+C_{\text { end }}
\end{equation}
それぞれのキャパシタンスはコンフォーマルマッピングを用いて解析的に求めることが可能である。それぞれの値を求めるにはまずC3について
\begin{equation}
    C_{3}=4 \varepsilon_{0} \varepsilon_{e} \frac{K\left(k_{1}\right)}{K\left(k_{1}^{\prime}\right)} L
\end{equation}\begin{equation}
    \varepsilon_{e}=1+q_{1} \frac{\varepsilon_{r}-1}{2}
\end{equation}
\begin{equation}
        q_{1}=\frac{K\left(k_{1}^{\prime}\right)}{K\left(k_{1}\right)} \frac{K\left(k_{2}\right)}{K\left(k_{2}^{\prime}\right)}
\end{equation}
\begin{equation}
    k_{1}=\frac{W}{W+2 S} \sqrt{\frac{1-\left[\frac{W+2 S}{3 W+2 S}\right]^{2}}{1-\left[\frac{W}{3 W+2 S}\right]^{2}}}
\end{equation}
\begin{equation}
    \begin{aligned}
    k_{2}=& \frac{\sinh \left(\frac{\pi W}{4 h}\right)}{\sinh \left(\frac{\pi(W+2 S)}{4 h}\right)} 
    & \sqrt{\frac{\sinh ^{2}\left(\frac{\pi(3 W+2 S)}{4 h}\right)-\sinh ^{2}\left(\frac{\pi(W+2 S)}{4 h}\right)}{\sinh ^{2}\left(\frac{\pi(3 W+2 S)}{4 h}\right)-\sinh ^{2}\left(\frac{\pi W}{4 h}\right)}}
    \end{aligned}
\end{equation}
同様にしてCNについて求めると
\begin{equation}
    C_{N}=(N-3) \varepsilon_{0} \varepsilon_{N} \frac{K\left(k_{3}\right)}{K\left(k_{3}^{\prime}\right)} L
\end{equation}
\begin{equation}
    \varepsilon_{N}=1+q_{N} \frac{\varepsilon_{r}-1}{2}
\end{equation}
\begin{equation}
    q_{N}=\frac{K\left(k_{3}^{\prime}\right)}{K\left(k_{3}\right)} \frac{K\left(k_{4}\right)}{K\left(k_{4}^{\prime}\right)},
\end{equation}
\begin{equation}
    k_{3}=\frac{W}{W+S},
\end{equation}
\begin{equation}
    \begin{aligned}
    k_{4}=& \frac{\sinh \left(\frac{\pi W}{4 h}\right)}{\sinh \left(\frac{\pi(W+ S)}{4 h}\right)} 
    & \sqrt{\frac{\sinh ^{2}\left(\frac{\pi(W+S)}{4 h}\right)+\sinh ^{2}\left(\frac{\pi(W+ S)}{4 h}\right)}{\cosh ^{2}\left(\frac{\pi(W)}{4 h}\right)-\sinh ^{2}\left(\frac{\pi (W+S)}{4 h}\right)}}
    \end{aligned}
\end{equation}
として求まる。最後にCendについて、この計算を行う上で主に参考にしている論文\cite*{Dib2005}ではCendの計算について\cite*{Dib2001ComprehensiveSO}中の式
\begin{equation}
    C_{o c}=c_{e f f} C_{o e}\left(\epsilon_{r}=1\right)
\end{equation}
\begin{equation}
    \label{Cend}
    C_{\alpha c}\left(c_{r}=1\right)=\frac{c_{0}}{\pi}\left[\frac{1}{g^{2}} f_{s}(g, W+S, 0)+\frac{1}{W^{2}} f_{0}(W, S, 0)\right]
\end{equation}
\begin{equation}
    \label{fs}
    \begin{aligned}
    f_{s}(a, b, c) &=\frac{4}{3} c^{3}+f(a, c)+f(b, c)-4 a b c \tan ^{-1}\left(\frac{a b}{c \tau}\right)-\frac{2}{3}\left(b^{2}-2 c^{2}+a^{2}\right) \tau \\
    &+\left(a^{2}-c^{2}\right) b \ln \left(\frac{\tau+b}{\tau-b}\right)+\left(b^{2}-c^{2}\right) a \ln \left(\frac{\tau+a}{\tau-a}\right)
    \end{aligned}
\end{equation}
\begin{equation}
    \label{f0}
    f_{0}(a, b, c) =\frac{4}{3} c^{3}+f(a, c)+f(a+b, c)-\frac{1}{2} f(b+2 a, c)-\frac{1}{2} f(b, c)
\end{equation}
\begin{equation}
    \label{fx}
    f(x, y) =\frac{2}{3}\left(x^{2}-2 y^{2}\right) \sqrt{x^{2}+y^{2}}+y^{2} x \ln \left(\frac{\sqrt{x^{2}+y^{2}}+x}{\sqrt{x^{2}+y^{2}}-x}\right)
\end{equation}
を使用したと記述されているが式\ref*{fx}の第2項目は不適切であるため、本稿では含めずに計算した。というのも式\ref*{fx}が使用されている式\ref{Cend,fs,f0}に注目すると式\label{fx}中の$y$は常にゼロであり、第2項目は2乗でゼロに収束する。よって2項目は計算に含まれないことになる。よって計算には以下
\begin{equation}
    \label{fx_re}
    f(x, y)_{revised} =\frac{2}{3}\left(x^{2}-2 y^{2}\right) \sqrt{x^{2}+y^{2}}
\end{equation}
をもちいた。
以上の計算式を用いてインターデジタルキャパシタンスの値を見積もることができる。
ここではさらに本稿で使用したパラメータを用いて櫛の本数Nに対してキャパシタンスの値がどれだけ変化するのか、また、実験結果との比較を行う。
\begin{figure}[H]
    \centering
    \includegraphics[width=10cm]{IDC_cal.pdf}
    \caption{キャパシタンス vs 櫛の本数}
\end{figure}
\section{ミアンダインダクタンス}
ミアンダインダクタンスとは図\ref{le}中蛇行した構造を持ったインダクタのことを指している。蛇行させることによりサイズに対して線路長が長くなり、また、相対する相互インダクタンスによりインダクタンスの総和が増加する。ここでは本稿で用いたミアンダインダクタンスの解析的計算方法について解説する。本文中ではここで行った解析的計算の結果と電磁界シミュレーションによるミアンダインダクタンス部分の計算を比較していので参照されたい。
\begin{figure}[H]
    \label{ミアンダ}
    \centering
    \includegraphics[width=14cm]{ミアンダ.pdf}
    \caption{ミアンダインダクタンス}
\end{figure}
以下数式は上図のパラメータに対応する。この計算に主に参考にした論文は\cite*{Stojanovic2004}である。
文献\cite*{Grover}より長方形線路の自己インダクタンスは
\begin{equation}
    L=0.002 l\{\ln [2 l /(w+t)]+0.50049+[(w+t) / 3 l]\}
\end{equation}
である。ここでwは線路幅、tは厚さ、lは線路長である。ミアンダインダクタンスの大部分を占めているのは各セグメントに於ける自己インダクタンスの総和である。すなわち、図中のパラメータによって各セグメントをラベル付けするとミアンダインダクタンス中セルフインダクタンスの寄与は
\begin{equation}
    L_{\text {selftot }}=2 \cdot L_{a}+2 \cdot L_{b}+N \cdot L_{h}+(N+1) \cdot L_{d}
\end{equation}
である。ここでNはセグメントhの本数に対応する。図\ref*{ミアンダ}ではN=10である。参考文献ではいくつかタイプミスが見受けられたため、少々煩雑ではあるが式の全文を記すこととする。
既に記した条件を前提とした上でNがの偶奇で計算は異なる。
\begin{figure}[H]
    \label{偶奇}
    \centering
    \includegraphics[width=14cm]{偶奇.pdf}
    \caption{ミアンダインダクタンス}
\end{figure}
文献\cite*{Grover}に依れば相対する線路に於ける相互インダクタンスは
\begin{equation}
    M=0.002 l\left[\log _{e}\left(\frac{l}{d}+\sqrt{1+\frac{l^{2}}{d^{2}}}\right)-\sqrt{1+\frac{d^{2}}{l^{2}}}+\frac{d}{l}\right]
\end{equation}
\begin{figure}[H]
    \label{相互}
    \centering
    \includegraphics[width=8cm]{相互.pdf}
    \caption{2線路間に於ける相互インダクタンス}
\end{figure}

\section{マスター方程式}
\section{2点相関関数}
